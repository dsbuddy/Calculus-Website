\documentclass[12pt]{article}


\usepackage{amssymb}
\usepackage{amsmath}
\usepackage{fullpage}
\usepackage{epsfig}
\usepackage{epstopdf}
\everymath{\displaystyle}
\usepackage{enumerate}



\begin{document}

\begin{center}
\underline{\LARGE{Chapter 4.2: Exponential \& Logarithmic Functions of Base $b$}}
\end{center}

\subsection*{Expected Skills:}

\begin{itemize}

\item Be able to specify the domain and range of $f(x)=b^x$ and $f(x)=\log_b{x}$.

\item Be able to graph $f(x)=b^x$ and $f(x)=\log_b{x}$, labeling all intersections with the coordinate axes and all asymptotes.

\item Be able to solve equations involving logarithm or exponential functions.

\item Be able to differentiate exponential and logarithmic functions; also, be able to solve application problems such as tangent line, rates of change, local/absolute extrema, and curve sketching.

\end{itemize}

\subsection*{Practice Problems: }

\begin{enumerate}

\item Evaluate each of the following without a calculator.

\begin{enumerate}

\item $\displaystyle \log_{4}{16}$

\includegraphics[scale=0.5]{start.pdf}
{2}
\includegraphics[scale=0.5]{end.pdf}


\item $\displaystyle \ln{\frac{1}{\sqrt[5]{e}}}$

\includegraphics[scale=0.5]{start.pdf}
{$\displaystyle -\frac{1}{5}$}
\includegraphics[scale=0.5]{end.pdf}


\item $\displaystyle \log_{43}{1}$

\includegraphics[scale=0.5]{start.pdf}
{0}
\includegraphics[scale=0.5]{end.pdf}


\item $\displaystyle \log_{16}{2}$

\includegraphics[scale=0.5]{start.pdf}
{$\displaystyle \frac{1}{4}$}
\includegraphics[scale=0.5]{end.pdf}


\end{enumerate}

\item Use the properties of logarithms to expand (as much as possible) the expression as a sum, difference, and/or constant multiple of logarithms.  (Assume that all variables are positive.)

\begin{enumerate}

\item $\displaystyle \log_{5}{(5x^{2}\sqrt{y})}$

\includegraphics[scale=0.5]{start.pdf}
{$\displaystyle 1+2\log_{5}{x}+\frac{1}{2}\log_{5}{y}$}
\includegraphics[scale=0.5]{end.pdf}


\item $\displaystyle \log_{6}{\frac{x^3}{y^2z^4}}$

\includegraphics[scale=0.5]{start.pdf}
{$\displaystyle 3\log_{6}{x}-2\log_{6}{y}-4\log_{6}{z}$}
\includegraphics[scale=0.5]{end.pdf}


\end{enumerate}

\item Determine the domain of the following functions.  Express your answer in interval notation.

\begin{enumerate}

\item $f(x)=\frac{\ln{(x-1)}}{x-5}$

\includegraphics[scale=0.5]{start.pdf}
{{ $(1,5)\cup(5,\infty)$}}
\includegraphics[scale=0.5]{end.pdf}


\item $f(x)=\frac{\sqrt{4-x}}{2^x-3}$

\includegraphics[scale=0.5]{start.pdf}
{{ $(-\infty,\log_2{3})\cup(\log_2{3},4]$}}
\includegraphics[scale=0.5]{end.pdf}


\item $f(x)=\frac{x-1}{2-\log_4{x}}$

\includegraphics[scale=0.5]{start.pdf}
{{ $(0,16)\cup(16,\infty)$}}
\includegraphics[scale=0.5]{end.pdf}


\end{enumerate}

\item Solve the given equation for $x$.  Where appropriate, you may leave your answers in logarithmic form.

\begin{enumerate}

\item $\displaystyle \left(3^{x-5}\right)-4=11$

\includegraphics[scale=0.5]{start.pdf}
{$\displaystyle x=\frac{\ln{15}}{\ln{3}}+5$}
\includegraphics[scale=0.5]{end.pdf}


\item $\displaystyle 2\log_{5}{(3x)}=4$

\includegraphics[scale=0.5]{start.pdf}
{$\displaystyle x=\frac{25}{3}$}
\includegraphics[scale=0.5]{end.pdf}


\item $\displaystyle \log_{3}{x}+\log_{3}{(x-8)}=2$

\includegraphics[scale=0.5]{start.pdf}
{$x=9$}
\includegraphics[scale=0.5]{end.pdf}


\item $\displaystyle \log_{8}{2x}+\log_{8}{(x+4)}=2$

\includegraphics[scale=0.5]{start.pdf}
{$x=4$}
\includegraphics[scale=0.5]{end.pdf}


\end{enumerate}

\item In a research experiment the population of a certain species is given by $P(t)=15(7^t)$, where $t$ is the number of weeks since the beginning of the experiment.

\begin{enumerate}

\item How large was the population at the beginning of the experiment?

\includegraphics[scale=0.5]{start.pdf}
{15}
\includegraphics[scale=0.5]{end.pdf}


\item How long will it take for the population to reach 300?  You may leave your answer in logarithmic form.

\includegraphics[scale=0.5]{start.pdf}
{$\displaystyle \frac{\ln{20}}{\ln{7}}$ weeks}
\includegraphics[scale=0.5]{end.pdf}


\end{enumerate}

\newpage

\item Calculate $\frac{dy}{dx}$.

\begin{enumerate}

\item $y = \log_{2}{(3x-1)}$ 

\includegraphics[scale=0.5]{start.pdf}
{{$\frac{3}{(3x-1)\ln2}$}}
\includegraphics[scale=0.5]{end.pdf}


\item $y=\frac{\log{x}}{2-\log{x}}$

\includegraphics[scale=0.5]{start.pdf}
{{$\frac{2}{x\ln{(10)}(2-\log{x})^2}$}}
\includegraphics[scale=0.5]{end.pdf}


\end{enumerate}

\item Use the change of base formula to express the following function in terms of the natural log.  Then, calculate $\frac{dy}{dx}$.  (Assume $x>0$.)

\begin{enumerate}

\item $y=\log_{x^2}{(e)}$

\includegraphics[scale=0.5]{start.pdf}
{{$y=\frac{1}{2\ln{x}}$; $\frac{dy}{dx}=-\frac{1}{2x(\ln{x})^2}$}}
\includegraphics[scale=0.5]{end.pdf}


\item $y = \log_{3x}{(x)}$ 

\includegraphics[scale=0.5]{start.pdf}
{{$y=\frac{\ln{x}}{\ln{3x}} $; $\frac{dy}{dx}=\frac{\ln{3}}{x(\ln{3x})^2}$}}
\includegraphics[scale=0.5]{end.pdf}


\end{enumerate}

\item Find an equation of the tangent line to the graph of $f(x)=3^{2x}$ at the point where $x=\log_3{4}$.

\includegraphics[scale=0.5]{start.pdf}
{$y-16=32(\ln{3})(x-\log_{3}4)$}
\includegraphics[scale=0.5]{end.pdf}


\end{enumerate}

\end{document}