\documentclass[12pt]{article}


\usepackage{amssymb}
\usepackage{amsmath}
\usepackage{fullpage}
\usepackage{epsfig}
\usepackage{epstopdf, xcolor, hyperref}
\everymath{\displaystyle}
\usepackage{enumerate}

\newif\ifans

\ansfalse

\begin{document}

\begin{center}
\underline{\LARGE{Tangent Planes \& Normal Lines}}
\end{center}

\noindent SUGGESTED REFERENCE MATERIAL:

\bigskip

\noindent As you work through the problems listed below, you should reference Chapter 13.7 of the recommended textbook (or the equivalent chapter in your alternative textbook/online resource) and your lecture notes.

\bigskip


\noindent EXPECTED SKILLS:

\begin{itemize}

\item Be able to compute an equation of the tangent plane at a point on the surface $z=f(x,y)$. 

\item Given an implicitly defined level surface $F(x,y,z)=k$, be able to compute an equation of the tangent plane at a point on the surface. 

\item Know how to compute the parametric equations (or vector equation) for the normal line to a surface at a specified point. 

\item Be able to use gradients to find tangent lines to the intersection curve of two surfaces. And, be able to find (acute) angles between tangent planes and other planes.

\end{itemize}

\noindent PRACTICE PROBLEMS:

\medskip

\noindent {\bf For problems 1-4, find two unit vectors which are normal to the given surface $S$ at the specified point $P$.}

\begin{enumerate}

\item $S: 2x-y+z=-7$; $P(-1,2,-3)$

\fbox{$\overrightarrow{n_{1,2}}=\pm \left\langle \frac{2}{\sqrt{6}},-\frac{1}{\sqrt{6}},\frac{1}{\sqrt{6}} \right\rangle$}

\item $S: x^2-3y+z^2=11$; $P(-1,-2,2)$

\fbox{$\overrightarrow{n_{1,2}}=\pm \left\langle -\frac{2}{\sqrt{29}},-\frac{3}{\sqrt{29}},\frac{4}{\sqrt{29}} \right\rangle$}

\item $S: z=y^4$; $P(3,-1,1)$

\fbox{$\overrightarrow{n_{1,2}}=\pm \left\langle 0,-\frac{4}{\sqrt{17}},-\frac{1}{\sqrt{17}} \right\rangle$}

\item $S: z=2-x^2\cos{(xy)}$; $P\left(-1,\frac{\pi}{2},2\right)$

\fbox{$\overrightarrow{n_{1,2}}=\pm \frac{2}{\sqrt{\pi^2+8}}\left\langle -\frac{\pi}{2},1,-1 \right\rangle$}

\end{enumerate}

\noindent {\bf For problems 5-9, compute equations of the tangent plane and the normal line to the given surface at the indicated point.}

\begin{enumerate}
\setcounter{enumi}{4}

\item $S: \ln{(x+y+z)}=2$; $P(-1,e^2,1)$

\fbox{$x+y+z=e^2$; $\overrightarrow{\ell}(t)=\langle -1,e^2,1\rangle +t\langle 1,1,1\rangle$}

\item $S: x^2+y^2+z^2=1$; $P\left(\frac{\sqrt{3}}{3}, \frac{\sqrt{3}}{3}, \frac{\sqrt{3}}{3} \right)$

\fbox{$x+y+z=\sqrt{3}$; $\overrightarrow{\ell}(t)=\left \langle \frac{\sqrt{3}}{3}, \frac{\sqrt{3}}{3}, \frac{\sqrt{3}}{3}\right \rangle +t\langle 1,1,1 \rangle$}

\item $S: z=\arcsin{\left(\frac{x}{y}\right)}$; $P\left(-1, -\sqrt{2}, \frac{\pi}{4}\right)$

\fbox{$-x+\frac{\sqrt{2}}{2}y-z=-\frac{\pi}{4}$; $\overrightarrow{\ell}(t)=\left\langle-1,-\sqrt{2},\frac{\pi}{4}\right\rangle+t\left\langle-1,\frac{\sqrt{2}}{2},-1\right\rangle$}

\item $S: x^2-xy+z^2=9$; $P(2,2,3)$

\fbox{$x-y+3z=9$; $\overrightarrow{\ell}(t)=\langle 2,2,3\rangle+t\langle1,-1,3\rangle$}

\item $S: z=x\cos{(x+y)}$; $P\left(\frac{\pi}{2}, \frac{\pi}{3}, -\frac{\sqrt{3}\pi}{4}\right)$

\fbox{\parbox{0.7\linewidth}{$(\pi+2\sqrt{3})\left(x-\frac{\pi}{2}\right)+\pi\left(y-\frac{\pi}{3}\right)+4\left(z+\frac{\sqrt{3}\pi}{4}\right)=0$\\
$\overrightarrow{\ell}(t)=\left\langle \frac{\pi}{2},\frac{\pi}{3},-\frac{\pi\sqrt{3}}{4}\right\rangle+t\left\langle \pi+2\sqrt{3},\pi,4\right\rangle$\\
\\
Detailed Solution: \textcolor{blue}{\href{http://www.math.drexel.edu/classes/Calculus/resources/Math200HW/Solutions/13_200_Tangent_09.pdf}{Here}}
}}

\item Consider the surfaces $S_1: x^2+y^2=25$ and $S_2: z=2-x$

\begin{enumerate}

\item Find an equation of the tangent line to the curve of intersection of $S_1$ and $S_2$ at the point $(3,4,-1)$.

\fbox{$\overrightarrow{\ell}(t)=\langle 3,4,-1\rangle+t\langle -4,3,4 \rangle$}

\item Find the acute angle between the planes which are tangent to the surfaces $S_1$ and $S_2$ at the point $(3,4,-1)$.

\fbox{$\pi-\cos^{-1}\left(\frac{-3}{5\sqrt{2}}\right)$}

\end{enumerate}

\item Consider the surfaces $S_1: z=x^2-y^2$ and $S_2: y^2+z^2=10$

\begin{enumerate}

\item Find an equation of the tangent line to the curve of intersection of $S_1$ and $S_2$ at the point $(2,1,3)$.

\fbox{$\overrightarrow{\ell}(t)=\langle 2,1,3\rangle+t\langle 5,12,-4 \rangle$; Detailed Solution: \textcolor{blue}{\href{http://www.math.drexel.edu/classes/Calculus/resources/Math200HW/Solutions/13_200_Tangent_11.pdf}{Here}}}

\item Find the acute angle between the planes which are tangent to the surfaces $S_1$ and $S_2$ at the point $(2,1,3)$.

\fbox{$\pi-\cos^{-1}\left(\frac{-10}{\sqrt{21}\sqrt{40}}\right)$; Detailed Solution: \textcolor{blue}{\href{http://www.math.drexel.edu/classes/Calculus/resources/Math200HW/Solutions/13_200_Tangent_11.pdf}{Here}}}

\end{enumerate}

\item Find all points on the ellipsoid $x^2+2y^2+3z^2=72$ where the tangent plane is parallel to the plane $4x+4y+12z=3$.

\fbox{$(4,2,4)$ and $(-4,-2,-4)$}

\item Find all points on the hyperboloid of 1 sheet $x^2+y^2-z^2=9$ where the normal line is parallel to the line which contains points $A(1,2,3)$ and $B(7,6,5)$.

\fbox{$\left(\frac{3\sqrt{3}}{2},\sqrt{3},-\frac{\sqrt{3}}{2}\right)$ and $\left(-\frac{3\sqrt{3}}{2},-\sqrt{3},\frac{\sqrt{3}}{2}\right)$; Detailed Solution: \textcolor{blue}{\href{http://www.math.drexel.edu/classes/Calculus/resources/Math200HW/Solutions/13_200_Tangent_13.pdf}{Here}}}

\item Two surfaces are called {\bf orthogonal} at a point of intersection if their normal lines are perpendicular at that point.  Show that the sphere $x^2+y^2+z^2=1$ and the cone $z^2=x^2+y^2$ are orthogonal at all points of intersection.  (HINT: Assume that the surfaces intersect at the arbitary point $(x_0,y_0,z_0)$.)

\fbox{\parbox{1\linewidth}{Suppose that $S_1: x^2+y^2+z^2=1$ and $S_2: z^2=x^2+y^2$ intersect at $(x_0,y_0,z_0)$. We will find a normal vector to each surface at the point $P_0$.  To do this, let $F(x,y,z)=x^2+y^2+z^2$ and $G(x,y,z)=x^2+y^2-z^2$.  Notice that $S_1$ is the level surface $F(x,y,z)=1$ and $S_2$ is the level surface $G(x,y,z)=0$.  So, $\nabla F(x_0,y_0,z_0)=\langle 2x_0,2y_0,2z_0\rangle$ and $\nabla G(x_0,y_0,z_0)=\langle 2x_0,2y_0,-2z_0\rangle$ are normal to $S_1$ and $S_2$, respectively, at the point $P_0$.  And, as a result, these vectors are parallel to the normal lines to $S_1$ and $S_2$ at $P_0$.\\
\\
Showing that the surfaces are orthogonal is equivalent to showing that $\nabla F(x_0,y_0,z_0) \perp \nabla G(x_0,y_0,z_0)$; i.e, $\nabla F(x_0,y_0,z_0) \cdot \nabla G(x_0,y_0,z_0)=0$.
\medskip
\begin{center}
(Continued on next page)
\end{center}
}}
\newpage
\fbox{\parbox{1\linewidth}{
\begin{align*}
\nabla F(x_0,y_0,z_0) \cdot \nabla G(x_0,y_0,z_0)&=\langle 2x_0,2y_0,2z_0\rangle\cdot\langle 2x_0,2y_0,-2z_0\rangle\\
&=4x_0^2+4y_0^2-4z_0^2\\
&=4(x_0^2+y_0^2-z_0^2)
\end{align*}
Since $(x_0,y_0,z_0)$ is a point of intersection of $S_1$ and $S_2$, it must satisfy both equations.  In particular, since it satisfies the equation for $S_2$, we have $x_0^2+y_0^2=z_0^2$.  Using this fact, we see that 
\begin{align*}
\nabla F(x_0,y_0,z_0) \cdot \nabla G(x_0,y_0,z_0)&=4(x_0^2+y_0^2-z_0^2)\\
&=4(z_0^2-z_0^2)\\
&=0
\end{align*}
As a result the surfaces are orthogonal to one another at the point of intersection, $(x_0,y_0,z_0)$.
}}

\item Show that every plane which is tangent to the cone $z^2=x^2+y^2$ must pass through the origin. (HINT: Compute the equation of the plane which is tangent to the surface at the point $P_0(x_0,y_0,z_0)$ and see what happens.)

\fbox{\parbox{1\linewidth}{Let $F(x,y,z)=x^2+y^2-z^2$.  The given surface is the level surface $F(x,y,z)=0$; so, $\nabla F(x_0,y_0,z_0)=\langle 2x_0,2y_0,-2z_0\rangle$ is normal to the given surface at the  point $(x_0,y_0,z_0)$.  Thus, an equation of the plane which is tangent to the given surface at the point $(x_0,y_0,z_0)$ is $2x_0(x-x_0)+2y_0(y-y_0)-2z_0(z-z_0)=0$; i.e., $x_0x+y_0y+z_0z-x_0^2-y_0^2+z_0^2=0$.\\
\\
Now, since $(x_0,y_0,z_0)$ is the point of tangency, it must also be a point on the surface.  Thus, $x_0^2+y_0^2=z_0^2 \Rightarrow -x_0^2-y_0^2=-z_0^2$.  Using this fact, the equation of the tangent plane can be written as:
\begin{align*}
x_0x+y_0y+z_0z-x_0^2-y_0^2+z_0^2&=0\\
x_0x+y_0y+z_0z-z_0^2+z_0^2&=0\\
x_0x+y_0y+z_0z&=0
\end{align*}
And, $(0,0,0)$ satisfies this equation.  Thus, since $(x_0,y_0,z_0)$ was an arbitrary point on the surface and its tangent plane passes through the origin, we have that all tangent planes to the surface must pass through the origin. 
}}

\end{enumerate}

\end{document}