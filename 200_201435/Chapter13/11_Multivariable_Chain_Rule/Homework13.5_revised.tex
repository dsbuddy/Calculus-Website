\documentclass[12pt]{article}


\usepackage{amssymb}
\usepackage{amsmath}
\usepackage{fullpage}
\usepackage{epsfig}
\usepackage{epstopdf}
\everymath{\displaystyle}
\usepackage{enumerate}



\begin{document}

\begin{center}
\underline{\LARGE{Multivariable Chain Rule}}
\end{center}

\noindent SUGGESTED REFERENCE MATERIAL:

\bigskip

\noindent As you work through the problems listed below, you should reference Chapter 13.5 of the recommended textbook (or the equivalent chapter in your alternative textbook/online resource) and your lecture notes.

\bigskip

\noindent EXPECTED SKILLS:

\begin{itemize}

\item Be able to compute partial derivatives with the various versions of the multivariate chain rule. 

\item Be able to compare your answer with the direct method of computing the partial derivatives.

\end{itemize}

\noindent PRACTICE PROBLEMS:

\medskip

\begin{enumerate}

\item Find $\frac{d z}{d t}$ by using the Chain Rule.  Check your answer by expressing $z$ as a function of $t$ and then differentiating. 
 
\begin{enumerate}

\item $z=2x-y,\ x=\sin{t},\ y=3t$ 

\includegraphics[scale=0.5]{start.pdf}
{{$\frac{dz}{dt}=2\cos{t}-3$}}
\includegraphics[scale=0.5]{end.pdf}


\item $z=x\sin{y},\ x=e^t,\ y=\pi t$ 

\includegraphics[scale=0.5]{start.pdf}
{{$\frac{dz}{dt}=e^{t}\sin{(\pi t)}+\pi e^{t}\cos{(\pi t)}$}}
\includegraphics[scale=0.5]{end.pdf}


\item $z=xy+y^2,\ x=t^2,\ y=t+1$ 

\includegraphics[scale=0.5]{start.pdf}
{{$\frac{dz}{dt}=3t^2+4t+2$}}
\includegraphics[scale=0.5]{end.pdf}


\item $z=\ln{\left(\frac{x^2}{y}\right)},\ x=e^{t},\ y=t^2$ 

\includegraphics[scale=0.5]{start.pdf}
{{$\frac{dz}{dt}=2-\frac{2}{t}$}}
\includegraphics[scale=0.5]{end.pdf}


\end{enumerate}

\item  Suppose $w=x^2+y^2+2z^2,\ x=t+1,\ y=\cos{t},\ z=\sin{t}$.   Find $\frac{d w}{d t}$ using the Chain Rule.  Check your answer by expressing $w$ as a function of $t$ and then differentiating.

\includegraphics[scale=0.5]{start.pdf}
{{$\frac{dw}{dt}=2t+2+2\sin{t}\cos{t}$}}
\includegraphics[scale=0.5]{end.pdf}


\item Suppose $f$ is a differentiable function of $x$ \& $y$, and define $g(u,v)=f(3u-v,u^2+v)$.  Use the table of values shown below to calculate $\left.\frac{\partial g}{\partial u}\right|_{(u,v)=(2,-1)}$ and $\left.\frac{\partial g}{\partial v}\right|_{(u,v)=(2,-1)}$.

\begin{center}
\begin{tabular}{|c|c|c|c|c|}
\hline
$(x,y)$& $f$ & $g$ & $f_x$ & $f_y$\\
\hline
$(2,-1)$ & 6 & $-7$ & 1 & 9\\
\hline
$(7,3)$ & 4 & 2 & $-3$ & 5\\
\hline
\end{tabular}
\end{center}

Hint: Decompose $f(3u-v,u^2+v)$ into $f(x,y)$ where $x=3u-v$ and $y=u^2+v$.

\includegraphics[scale=0.5]{start.pdf}
{{$g_u(2,-1)=11$; $g_v(2,-1)=8$}}
\includegraphics[scale=0.5]{end.pdf}


\newpage

\item Find $\frac{\partial w}{\partial s}$ and $\frac{\partial w}{\partial t}$ by using the appropriate Chain Rule. 

\begin{enumerate}

\item $w=xy\sin{\left(z^2\right)},\ x=s-t,\ y=s^2,\ z=t^2$ 

\includegraphics[scale=0.5]{start.pdf}
{{$\frac{\partial w}{\partial s}=s^2\sin{\left(t^4\right)}+2s(s-t)\sin{\left(t^4\right)}$; $\frac{\partial w}{\partial t}=-s^2\sin{\left(t^4\right)}+4s^2t^3(s-t)\cos{\left(t^4\right)}$}}
\includegraphics[scale=0.5]{end.pdf}


\item $w=xy+yz,\ x=s+t,\ y=st,\ z=s-2t$ 

\includegraphics[scale=0.5]{start.pdf}
{{$\frac{\partial w}{\partial s}=4st-t^2$; $\frac{\partial w}{\partial t}=2s^2-2st$}}
\includegraphics[scale=0.5]{end.pdf}


\end{enumerate}

\item Suppose that $J=f(x,y,z,w)$, where $x=x(r,s,t)$, $y=y(r,t)$, $z=z(r,s)$ and $w=w(s,t)$.  Use the Chain Rule to find $\frac{\partial J}{\partial r}$, $\frac{\partial J}{\partial s}$, and $\frac{\partial J}{\partial t}$.

\includegraphics[scale=0.5]{start.pdf}
{{{0.4\linewidth}{$\frac{\partial J}{\partial r}=\frac{\partial f}{\partial x}\frac{\partial x}{\partial r}+\frac{\partial f}{\partial y}\frac{\partial y}{\partial r}+\frac{\partial f}{\partial z}\frac{\partial z}{\partial r}$; \\
\\
$\frac{\partial J}{\partial s}=\frac{\partial f}{\partial x}\frac{\partial x}{\partial s}+\frac{\partial f}{\partial z}\frac{\partial z}{\partial s}+\frac{\partial f}{\partial w}\frac{\partial w}{\partial s}$;\\
\\
$\frac{\partial J}{\partial t}=\frac{\partial f}{\partial x}\frac{\partial x}{\partial t}+\frac{\partial f}{\partial y}\frac{\partial y}{\partial t}+\frac{\partial f}{\partial w}\frac{\partial w}{\partial t}$
}}}
\includegraphics[scale=0.5]{end.pdf}


\item Suppose $g=f(u-v,v-w,w-u)$.  Show that $\frac{\partial g}{\partial u}+\frac{\partial g}{\partial v}+\frac{\partial g}{\partial w}=0$.

\includegraphics[scale=0.5]{start.pdf}
{{{1\linewidth}{We can express $g$ as $g=f(x,y,z)$, where $x=u-v$, $y=v-w$, and $z=w-u$.  Then, we compute the partial derivatives of $g$ with respect to $u$, $v$, and $w$.

$$\frac{\partial g}{\partial u}=f_x(u-v,v-w,w-u)-f_z(u-v,v-w,w-u)$$
$$\frac{\partial g}{\partial v}=-f_x(u-v,v-w,w-u)+f_y(u-v,v-w,w-u)$$
$$\frac{\partial g}{\partial w}=-f_y(u-v,v-w,w-u)+f_z(u-v,v-w,w-u)$$

Summing these three partial derivatives yields 0.}}}
\includegraphics[scale=0.5]{end.pdf}


\newpage

\item Suppose $u=u(x,y)$, $v=v(x,y)$, $x=r\cos{\theta}$, and $y=r\sin{\theta}$.

\begin{enumerate}

\item Calculate $\frac{\partial u}{\partial r}$, $\frac{\partial u}{\partial \theta}$, $\frac{\partial v}{\partial r}$, and $\frac{\partial v}{\partial \theta}$

\includegraphics[scale=0.5]{start.pdf}
{{{0.7\linewidth}{$\frac{\partial u}{\partial r}=\frac{\partial u}{\partial x}\cos{\theta}+\frac{\partial u}{\partial y}\sin{\theta}$; $\frac{\partial u}{\partial \theta}=-r\frac{\partial u}{\partial x}\sin{\theta}+r\frac{\partial u}{\partial y}\cos{\theta}$;\\
\\
$\frac{\partial v}{\partial r}=\frac{\partial v}{\partial x}\cos{\theta}+\frac{\partial v}{\partial y}\sin{\theta}$; $\frac{\partial v}{\partial \theta}=-r\frac{\partial v}{\partial x}\sin{\theta}+r\frac{\partial v}{\partial y}\cos{\theta}$}}}
\includegraphics[scale=0.5]{end.pdf}


\item Suppose that $u(x,y)$ and $v(x,y)$ satisfy the Cauchy-Riemann Equations:

$$\frac{\partial u}{\partial x}=\frac{\partial v}{\partial y}$$
$$\frac{\partial u}{\partial y}=-\frac{\partial v}{\partial x}$$

Use this along with part (a) to derive the polar form of the Cauchy-Riemann Equations:

$$\frac{\partial u}{\partial r}=\frac{1}{r}\frac{\partial v}{\partial \theta}$$
$$\frac{\partial u}{\partial \theta}=-r\frac{\partial v}{\partial r}$$

\includegraphics[scale=0.5]{start.pdf}
{{{1\linewidth}{From part (a), we know that $\frac{\partial u}{\partial r}=\frac{\partial u}{\partial x}\cos{\theta}+\frac{\partial u}{\partial y}\sin{\theta}$.  We apply the Cauchy Riemann Equations to $\frac{\partial u}{\partial x}$ and $\frac{\partial u}{\partial y}$.  Thus, 
\begin{align*}
\frac{\partial u}{\partial r}&=\frac{\partial u}{\partial x}\cos{\theta}+\frac{\partial u}{\partial y}\sin{\theta}\\
&=\frac{\partial v}{\partial y}\cos{\theta}-\frac{\partial v}{\partial x}\sin{\theta}\\
&=\frac{1}{r}\left(r\frac{\partial v}{\partial y}\cos{\theta}-r\frac{\partial v}{\partial x}\sin{\theta}\right)\\
&=\frac{1}{r}\frac{\partial v}{\partial \theta}
\end{align*}
A similar argument yields the second Cauchy Riemann Equation in polar coordinates.
}}}
\includegraphics[scale=0.5]{end.pdf}




\end{enumerate}

\end{enumerate}

\end{document}