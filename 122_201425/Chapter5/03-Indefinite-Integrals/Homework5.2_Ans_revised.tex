\documentclass[12pt]{article}


\usepackage{amssymb}
\usepackage{amsmath}
\usepackage{fullpage}
\usepackage{epsfig}
\usepackage{epstopdf, hyperref,xcolor}
\everymath{\displaystyle}



\begin{document}

\begin{center}
\underline{\LARGE{The Indefinite Integral}}
\end{center}

\noindent SUGGESTED REFERENCE MATERIAL:

\bigskip

\noindent As you work through the problems listed below, you should reference Chapter 5.2 of the recommended textbook (or the equivalent chapter in your alternative textbook/online resource) and your lecture notes.

\bigskip


\noindent EXPECTED SKILLS:

\begin{itemize}

\item Given a differentiation rule, be able to construct the associated indefinite integration rule. 

\item Know how to integrate power functions (including polynomials), exponential functions, \& trigonometric functions.

\end{itemize}

\noindent PRACTICE PROBLEMS:

\medskip

\noindent {\bf For problems 1 and 2, compute the indicated derivative and state a corresponding integration formula.}

\begin{enumerate}

\item $\frac{d}{dx}\left[\frac{1}{(2x+3)^2}\right]$ 

\includegraphics[scale=0.5]{start.pdf}
{{$\frac{d}{dx}\left[\frac{1}{(2x+3)^2}\right]=\frac{-4}{(2x+3)^3} \implies \int \frac{-4}{(2x+3)^3} \,dx= \frac{1}{(2x+3)^2}+C$}}
\includegraphics[scale=0.5]{end.pdf}


\item $\frac{d}{dx} [x\ln{x}-x]$ 

\includegraphics[scale=0.5]{start.pdf}
{{$\frac{d}{dx} [x\ln{x}-x]=\ln{x} \implies \int \ln{x} \,dx = x\ln{x}-x+C$}}
\includegraphics[scale=0.5]{end.pdf}


\end{enumerate}

\noindent {\bf For problems 3-18, evaluate given indefinite integral and check your answer by differentiation.}

\begin{enumerate}
\setcounter{enumi}{2}

\item $\int \left(\frac{1}{2}x+x^2 \right)\,dx$ 

\includegraphics[scale=0.5]{start.pdf}
{{$\frac{1}{4}x^2+\frac{1}{3}x^3+C$}}
\includegraphics[scale=0.5]{end.pdf}


\item $\int \left(\sqrt{x^7}+e\right) \,dx$

\includegraphics[scale=0.5]{start.pdf}
{{$\frac{2}{9}x^{9/2}+ex+C$}}
\includegraphics[scale=0.5]{end.pdf}


\item $\int \left(\frac{1}{x^3}+3x^3\right) \,dx$ 

\includegraphics[scale=0.5]{start.pdf}
{{$\frac{-1}{2}x^{-2}+\frac{3}{4}x^4+C$}}
\includegraphics[scale=0.5]{end.pdf}


\item $\int \left(3x^{-2/3}+x^{-1/2} + 5x\right) \,dx$ 

\includegraphics[scale=0.5]{start.pdf}
{{$9x^{1/3}+2x^{1/2}+\frac{5}{2}x^2+C$}}
\includegraphics[scale=0.5]{end.pdf}


\item $\int \left(4x^{4/3}-7\sqrt{x}\right) \,dx$ 

\includegraphics[scale=0.5]{start.pdf}
{{$\frac{12}{7}x^{7/3}-\frac{14}{3}x^{3/2}+C$}}
\includegraphics[scale=0.5]{end.pdf}


\item $\int 3\cos{x} \,dx$ 

\includegraphics[scale=0.5]{start.pdf}
{{$3\sin{x}+C$}}
\includegraphics[scale=0.5]{end.pdf}


\item $\int -7\sec^{2}{x} \,dx$ 

\includegraphics[scale=0.5]{start.pdf}
{{$-7\tan{x}+C$}}
\includegraphics[scale=0.5]{end.pdf}


\item $\int \left(-\frac{1}{x}+ e^{x}\right) \,dx$ 

\includegraphics[scale=0.5]{start.pdf}
{{$-\ln{|x|}+e^{x}+C$}}
\includegraphics[scale=0.5]{end.pdf}


\item $\int (1-x^2)(x^3+4) \,dx$ 

\includegraphics[scale=0.5]{start.pdf}
{{$-\frac{1}{6}x^6+\frac{1}{4}x^4-\frac{4}{3}x^3+4x+C$}}
\includegraphics[scale=0.5]{end.pdf}


\item $\int \frac{x^2-3x^5}{x^3} \,dx$.

\includegraphics[scale=0.5]{start.pdf}
{{$\ln{|x|}-x^3+C$; Video Solution: \textcolor{blue}{\href{https://www.youtube.com/watch?v=JCbEFor0NYY}{https://www.youtube.com/watch?v=JCbEFor0NYY}}}}
\includegraphics[scale=0.5]{end.pdf}


\item $\int \frac{-2\sin{x}}{\cos^{2}x} \,dx$ 

\includegraphics[scale=0.5]{start.pdf}
{{$-2\sec{x}+C$}}
\includegraphics[scale=0.5]{end.pdf}


\item $\int \frac{1}{\sqrt{4-4x^2}} \,dx$ 

\includegraphics[scale=0.5]{start.pdf}
{{$\frac{1}{2}\sin^{-1}{x}+C$; Detailed Solution: \textcolor{blue}{\href{http://www.math.drexel.edu/classes/Calculus/resources/Math122HW/Solutions/122_03_Antiderivatives_14.pdf}{Here}}}}
\includegraphics[scale=0.5]{end.pdf}


\item $\int \left(6\cos{x}+9\csc^{2}{x}\right) \,dx$ 

\includegraphics[scale=0.5]{start.pdf}
{{$6\sin{x}-9\cot{x}+C$}}
\includegraphics[scale=0.5]{end.pdf}


\item $\int \left(\sin{x}-3\sec{x}\tan{x}\right) \,dx$ 

\includegraphics[scale=0.5]{start.pdf}
{{$-\cos{x}-3\sec{x}+C$}}
\includegraphics[scale=0.5]{end.pdf}


\item $\int 2^{x} \,dx$ 

\includegraphics[scale=0.5]{start.pdf}
{{$\frac{2^{x}}{\ln{2}}+C$}}
\includegraphics[scale=0.5]{end.pdf}


\item $\int \frac{x^2}{x^2+1} \,dx$ (HINT: Use polynomial division)

\includegraphics[scale=0.5]{start.pdf}
{{$x-\arctan{x}+C$; Video Solution: \textcolor{blue}{\href{https://www.youtube.com/watch?v=kTLkMO8l2Ak}{https://www.youtube.com/watch?v=kTLkMO8l2Ak}}}}
\includegraphics[scale=0.5]{end.pdf}


\item Consider $\int{\cot^{2}{x}} \,dx$.

\begin{enumerate}

\item Using the fact that $\sin^{2}{x}+\cos^{2}{x}=1$, derive the identity $\cot^2{x}=\csc^{2}{x}-1$.

\includegraphics[scale=0.5]{start.pdf}
{{{1\linewidth}{ Provided that $x \neq \pi \cdot k$, where $k$ is any integer, we have:
\begin{align*} 
\sin^{2}{x}+\cos^{2}{x} &=1\\
\frac{\sin^{2}{x}}{\sin^{2}{x}}+\frac{\cos^{2}{x}}{\sin^2{x}}&=\frac{1}{\sin^2{x}}\\
1+\cot^2{x}&=\csc^2{x}\\
\cot^2{x}&=\csc^2{x}-1
\end{align*}
}}}
\includegraphics[scale=0.5]{end.pdf}


\item Use the identity that you derived in part (a) to evaluate the original integral.

\includegraphics[scale=0.5]{start.pdf}
{{$\int{\cot^{2}{x}} \,dx=-\cot{x}-x+C$.}}
\includegraphics[scale=0.5]{end.pdf}


\end{enumerate}

\end{enumerate}

\newpage

\noindent {\bf For problems 20 and 21, find a function $y=y(x)$ which satisfies the given Initial Value Problem.}

\begin{enumerate}
\setcounter{enumi}{19}

\item $\left\{\begin{array}{l}
\frac{dy}{dx}=\frac{1}{9x^2}\\
\\
y(1)=\frac{1}{2}
\end{array}\right.$

\includegraphics[scale=0.5]{start.pdf}
{{$y=-\frac{1}{9}x^{-1}+\frac{11}{18}$}}
\includegraphics[scale=0.5]{end.pdf}


\item $\left\{\begin{array}{l}
\frac{dy}{dx}=-2e^{x}\\
\\
y(0)=-5
\end{array}\right.$

\includegraphics[scale=0.5]{start.pdf}
{{$y=-2e^{x}-3$; Video Solution: \textcolor{blue}{\href{https://www.youtube.com/watch?v=kvFRPT4nTIM}{https://www.youtube.com/watch?v=kvFRPT4nTIM}}}}
\includegraphics[scale=0.5]{end.pdf}


\item A ball is thrown straight up in the air from an initial height of $s_0$ feet above the ground with an initial speed of $v_0$ ft/sec.  Then $s(t)$ gives the height (in feet) above the ground at time $t$, $v(t)=s^{\prime}(t)$ gives the velocity (in ft/sec) of the ball at time $t$, and $a(t)=s^{\prime \prime}(t)$ gives the acceleration (in $\text{ft/sec}^2$) of the ball at time $t$.  Assuming that acceleration is constant, $-g \text{ ft/sec}^2$, determine $v(t)$ and $s(t)$.

\includegraphics[scale=0.5]{start.pdf}
{{$v(t)=-gt+v_0$, $s(t)=-\frac{1}{2}gt^2+v_0t+s_0$}}
\includegraphics[scale=0.5]{end.pdf}


\end{enumerate}

\end{document}