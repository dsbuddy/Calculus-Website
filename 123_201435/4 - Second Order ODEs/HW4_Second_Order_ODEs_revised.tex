\documentclass[12pt]{article}


\usepackage{amssymb}
\usepackage{amsmath}
\usepackage{fullpage}
\usepackage{epsfig}
\usepackage{epstopdf}
\everymath{\displaystyle}
\usepackage{enumerate}
\usepackage[hidelinks]{hyperref}
\usepackage{enumitem}
\usepackage[hidelinks]{hyperref}
\usepackage{xcolor}




\begin{document}

\begin{center}
\underline{\LARGE{Second-Order Linear Homogeneous ODE's}}
\end{center}

\noindent SUGGESTED REFERENCE MATERIAL:

\medskip

\noindent As you work through the problems listed below, you should reference your lecture notes and the relevant chapters in a textbook/online resource.

\bigskip

\noindent EXPECTED SKILLS:

\medskip

\begin{itemize}[topsep=0pt]

\item Show that two functions are linearly independent.

\item Be able to solve second-order linear ODE's with constant coefficients by using the appropriate auxiliary equations.

\item Be able to solve initial value problems and boundary value problems involving second-order linear ODE's with constant coefficients.

\end{itemize}

\bigskip

\noindent PRACTICE PROBLEMS:

\begin{enumerate}

\item Suppose that $y_1(x)$ and $y_2(x)$ are solutions to the second-order ODE $$P(x)\frac{d^2y}{dx^2}+Q(x)\frac{dy}{dx}+R(x)y=0.$$  Show that $y(x)=c_1y_1(x)+c_2y_2(x)$ is also a solution, where $c_1$ and $c_2$ are arbitrary constants.

\includegraphics[scale=0.5]{start.pdf}
{{{1\linewidth}{To simplify notation, we will omit the ``of $x$" from the functions $y, y_1,$ and $y_2$ and also write derivatives in prime notation.  Now, since $y_1$ and $y_2$ are solutions
to the ODE, they satisfy it, i.e. 
\begin{equation}P(x)y_1''+Q(x)y_1'+R(x)y_1 = 0 \text{ and } P(x)y_2''+Q(x)y_2'+R(x)y_2 = 0. \tag{*} 
\end{equation}  Now we show that $y=c_1y_1+c_2y_2$ also satisfies the ODE: 
\begin{align*} P(x)y''+Q(x)y'+R(x)y&=P(x)(c_1y_1+c_2y_2)''+Q(x)(c_1y_1+c_2y_2)' + R(x)(c_1y_1+c_2y_2) \\
&=P(x)(c_1y_1''+c_2y_2'')+Q(x)(c_1y_1'+c_2y_2') + R(x)(c_1y_1+c_2y_2) \\
&=c_1\left[P(x)y_1''+Q(x)y_1'+R(x)y_1\right] +c_2\left[P(x)y_2''+Q(x)y_2'+R(x)y_2\right] \\
&=c_1(0) + c_2(0) \text{ by (*) } \\
&=0. \end{align*} Thus $y=c_1y_1+c_2y_2$ is a solution to the ODE. }}}
\includegraphics[scale=0.5]{end.pdf}


\item Consider the functions $y_1=e^{m_1x}$ and $y_2=e^{m_2x}$, where $m_1$ and $m_2$ are arbitrary real numbers with $m_1 \neq m_2$.  Show that the functions are linearly independent.

\includegraphics[scale=0.5]{start.pdf}
{{{1\linewidth}{If $y_1$ and $y_2$ were not linearly independent, then one would be a constant multiple of the other, i.e. $e^{m_1x}=ke^{m_2x}$ for {\bf all} $x$.  In particular, for $x=0$
we would have $e^{(m_1)(0)}=ke^{(m_2)(0)}$, or $1=k$.  But then  $e^{m_1x}=ke^{m_2x}$ becomes $e^{m_1x}=e^{m_2x}$, which is impossible since $m_1 \neq m_2$.  Thus, $y_1=e^{m_1x}$ and $y_2=e^{m_2x}$ are linearly independent.}}}
\includegraphics[scale=0.5]{end.pdf}


\item Consider the functions $y_1=e^{\alpha x}\cos\left({\beta x}\right)$ and $y_2=e^{\alpha x}\sin\left({\beta x}\right)$, where $\alpha$ and $\beta$ are arbitrary real numbers.  
Show that the functions are linearly independent.

\includegraphics[scale=0.5]{start.pdf}
{{{1\linewidth}{If $y_1$ and $y_2$ were not linearly independent, then one would be a constant multiple of the other, i.e. 
$e^{\alpha x}\cos\left({\beta x}\right)=ke^{\alpha x}\sin\left({\beta x}\right)$ for {\bf all} $x$.  In particular, for $x=0$
we would have $0=1$, which is absurd.  Thus, $y_1=e^{\alpha x}\cos\left({\beta x}\right)$ and $y_2=e^{\alpha x}\sin\left({\beta x}\right)$ are linearly independent.}}}
\includegraphics[scale=0.5]{end.pdf}


\end{enumerate}

\noindent {\bf For problems 4 -- 10, solve the second-order differential equation.  Express your answer as an explicit function of $x$.}
 
\begin{enumerate}
\setcounter{enumi}{3}
  
\item $y''+5y'+4y=0$

\includegraphics[scale=0.5]{start.pdf}
{{$y=c_1e^{-4x}+c_2e^{-x}$}}
\includegraphics[scale=0.5]{end.pdf}


\item $25y=10\frac{dy}{dx}-\frac{d^2y}{dx^2}$

\includegraphics[scale=0.5]{start.pdf}
{{$y=c_1e^{5x}+c_2xe^{5x}$}}
\includegraphics[scale=0.5]{end.pdf}


\item $2\frac{d^2y}{dx^2}+5\frac{dy}{dx}-3y=0$

\includegraphics[scale=0.5]{start.pdf}
{{$y=c_1e^{x/2}+c_2e^{-3x}$}}
\includegraphics[scale=0.5]{end.pdf}


\item $y''+2y'+3y=0$

\includegraphics[scale=0.5]{start.pdf}
{{$y=c_1e^{-x}\cos\left(\sqrt{2}x\right)+c_2e^{-x}\sin\left(\sqrt{2}x\right)$}}
\includegraphics[scale=0.5]{end.pdf}


\item $9y''+12y'+4y=0$

\includegraphics[scale=0.5]{start.pdf}
{{$y=c_1e^{-2x/3}+c_2xe^{-2x/3}$}}
\includegraphics[scale=0.5]{end.pdf}


\item $2\frac{d^2y}{dx^2}-6\frac{dy}{dx}+17y=0$

\includegraphics[scale=0.5]{start.pdf}
{{$y=c_1e^{3x/2}\cos\left(\frac{5}{2}x\right)+c_2e^{3x/2}\sin\left(\frac{5}{2}x\right)$; Detailed Solution: \textcolor{blue}{\href{http://www.math.drexel.edu/classes/Calculus/resources/Math123HW/Solutions/123_04_Second_Order_ODEs_09.pdf}{Here}}}}
\includegraphics[scale=0.5]{end.pdf}



\item $9y''=4y$

\includegraphics[scale=0.5]{start.pdf}
{{$y=c_1e^{2x/3}+c_2e^{-2x/3}$}}
\includegraphics[scale=0.5]{end.pdf}


\item Solve the initial value problem. \newline \newline
$\left\{\begin{array}{l}
y''+5y'+4y=0
\\ \\
y(0)=5
\\ \\
y'(0)=7
\end{array}\right.$

\includegraphics[scale=0.5]{start.pdf}
{{$y=-4e^{-4x}+9e^{-x}$}}
\includegraphics[scale=0.5]{end.pdf}


\item Consider a curve $y=y(x)$ that goes through the point $\left(\frac{\pi}{\sqrt{2}},e^{-\pi/\sqrt{2}}\right)$, has a horizontal tangent line at that point, and satisfies the ODE 
$y''+2y'+3y=0$.

\begin{enumerate}

\item Set up an initial value problem (IVP) whose solution is $y(x)$.

\includegraphics[scale=0.5]{start.pdf}
{{$\left\{\begin{array}{l}
y''+2y'+3y=0
\\ \\
y\left(\frac{\pi}{\sqrt{2}}\right)=e^{-\pi/\sqrt{2}}
\\ \\
y'\left(\frac{\pi}{\sqrt{2}}\right)=0
\end{array}\right.$}}
\includegraphics[scale=0.5]{end.pdf}


\item Solve the IVP from part (a).

\includegraphics[scale=0.5]{start.pdf}
{{$y=-e^{-x}\cos\left(\sqrt{2}x\right)-\frac{1}{\sqrt{2}}e^{-x}\sin\left(\sqrt{2}x\right)$}}
\includegraphics[scale=0.5]{end.pdf}


\end{enumerate}

\item Consider a curve $y=y(x)$ that has a $y$-intercept of $(0,1)$, has a tangent line at the $y$-intercept that is perpendicular to the line $y=-\frac{1}{2}x+999$, and satisfies the ODE 
$9y''=4y$.

\begin{enumerate}

\item Set up an initial value problem (IVP) whose solution is $y(x)$.

\includegraphics[scale=0.5]{start.pdf}
{{$\left\{\begin{array}{l}
9y''=4y
\\ \\
y(0)=1
\\ \\
y'(0)=2
\end{array}\right.$}}
\includegraphics[scale=0.5]{end.pdf}


\item Solve the IVP from part (a).

\includegraphics[scale=0.5]{start.pdf}
{{$y=2e^{2x/3}-e^{-2x/3}$}}
\includegraphics[scale=0.5]{end.pdf}


\end{enumerate}

\item Consider a curve $y=y(x)$ that goes through both the origin and the point $\left(\ln{2},64\right)$ and also satisfies the ODE $25y=10\frac{dy}{dx}-\frac{d^2y}{dx^2}$.

\begin{enumerate}

\item A \underline{boundary value problem}, or BVP, is a problem that consists of a differential equation and information about two \emph{different} points on the solution curve.  \newline 
[Whereas an initial value problem, or IVP, consists of a differential equation and initial conditions about a \emph{single} point.] \newline
Set up a BVP whose solution is $y(x)$.

\includegraphics[scale=0.5]{start.pdf}
{{$\left\{\begin{array}{l}
 25y=10\frac{dy}{dx}-\frac{d^2y}{dx^2}
\\ \\
y(0)=0
\\ \\
y(\ln{2})=64
\end{array}\right.$; Detailed Solution: \textcolor{blue}{\href{http://www.math.drexel.edu/classes/Calculus/resources/Math123HW/Solutions/123_04_Second_Order_ODEs_14.pdf}{Here}}}}
\includegraphics[scale=0.5]{end.pdf}


\item Solve the BVP from part (a).

\includegraphics[scale=0.5]{start.pdf}
{{$y=\frac{2}{\ln{2}}xe^{5x}$}}
\includegraphics[scale=0.5]{end.pdf}


\end{enumerate}

\item Suppose an object of mass $M$ is attached to a spring, and the spring is allowed to hang at its natural length.  Let $x=x(t)$ represent the position of the object when the spring is stretched (or compressed)
$x$ units from its natural length, with $x=0$ corresponding to the spring's natural length.  If we ignore any forces acting on the object except it's weight and the restoring force of the spring, then based on Newton's
Second Law and Hooke's Law it can be shown that $$M\frac{d^2x}{dt^2}+kx=0$$ where $k$ is a some positive constant, called the \underline{spring constant}.

\begin{enumerate}

\item Solve the above second-order ODE.

\includegraphics[scale=0.5]{start.pdf}
{{$x(t)=c_1\cos{\left(\sqrt{\frac{k}{M}}t\right)}+c_2\sin{\left(\sqrt{\frac{k}{M}}t\right)}$; Detailed Solution: \textcolor{blue}{\href{http://www.math.drexel.edu/classes/Calculus/resources/Math123HW/Solutions/123_04_Second_Order_ODEs_15.pdf}{Here}}}}
\includegraphics[scale=0.5]{end.pdf}


\item Suppose that an object of mass $3$ kg is attached to a spring with spring constant $75$ whose natural length is $1$ meter.  If the spring is stretched to a length of $1.5$ meters and is released with zero initial velocity, find the position of the object at any time $t$.  Describe the motion of the object over time.

\includegraphics[scale=0.5]{start.pdf}
{{{1\linewidth}{The position of the object at any time $t$ is $x(t)=0.5\cos{\left(5t\right)}$, so the spring will continue to oscillate indefinitely.  This type of motion
is called \underline{simple harmonic motion}.  To see an animation of this, visit the Wikipedia page: \\ \url{http://upload.wikimedia.org/wikipedia/commons/2/25/Animated-mass-spring.gif}
 }}}
\includegraphics[scale=0.5]{end.pdf}
   

 \end{enumerate}

\end{enumerate}

\end{document}