\documentclass[12pt]{article}


\usepackage{amssymb}
\usepackage{amsmath}
\usepackage{fullpage}
%\usepackage{epsfig}
%\usepackage{epstopdf}
\everymath{\displaystyle}
\usepackage{enumerate}
\usepackage{enumitem}
\usepackage[hidelinks]{hyperref}
\usepackage{xcolor}

\newif\ifans

\anstrue

\begin{document}

\begin{center}
\underline{\LARGE{Differentiating and Integrating Power Series}}
\end{center}

\noindent SUGGESTED REFERENCE MATERIAL:

\medskip

\noindent As you work through the problems listed below, you should reference your lecture notes and the relevant chapters in a textbook/online resource.

\medskip

\noindent EXPECTED SKILLS:

\medskip

\begin{itemize}[topsep=0pt]

\item Know (i.e. memorize) the Maclaurin series for $e^x, \sin{x}$ and $\cos{x}$.  Algebraically manipulate these series expansions, as well as other given power series expansions, to form new expansions.

\item Differentiate and integrate power series expansions term-by-term.

\item Use a series expansion to approximate an integral to some specified accuracy.   

\end{itemize}

\medskip

\noindent PRACTICE PROBLEMS:

\medskip

\begin{enumerate}

\item Confirm that $\frac{d}{dx}(e^x)=e^x$ by differentiating the Maclaurin series for $e^x$ term-by-term.

\ifans{\fbox{\parbox{1\linewidth}{\begin{align*} \frac{d}{dx}(e^x)&=\frac{d}{dx}\left[1+x+\frac{x^2}{2!}+\frac{x^3}{3!}+\frac{x^4}{4!}+\ldots \right] \\
&=0+1+\frac{2x}{2!}+\frac{3x^2}{3!}+\frac{4x^3}{4!}+\ldots \\
&=1+x+\frac{x^2}{2!}+\frac{x^3}{3!}+\ldots \\
&=e^x. \end{align*}  }}} \fi

\item Recall that the Maclaurin series for $e^{x}$ converges to $e^{x}$ for all real numbers $x$.  It can be shown (in a complex analysis course) that this convergence holds for any complex number as well.  Based on this fact, use the Maclaurin
series for $e^x, \sin{x},$ and $\cos{x}$ to prove \underline{Euler's Formula}: $$e^{ix}=\cos{x}+i\sin{x},$$ where $i$ is the imaginary number with the property $i^2=-1$ (and thus $i^3=-i$, $i^4=1$, etc).

\ifans{\fbox{\parbox{1\linewidth}{Since $e^x=1+x+\frac{x^2}{2!}+\frac{x^3}{3!}+\ldots$ \text{ }, we have 
\begin{align*}  e^{ix}&=1+ix+\frac{(ix)^2}{2!}+\frac{(ix)^3}{3!}+\frac{(ix)^4}{4!}+\frac{(ix)^5}{5!}+\frac{(ix)^6}{6!}+\frac{(ix)^7}{7!}+\frac{(ix)^8}{8!}+\ldots
\\ &=1+ix-\frac{x^2}{2!}-i\frac{x^3}{3!}+\frac{x^4}{4!}+i\frac{x^5}{5!}-\frac{x^6}{6!}-i\frac{x^7}{7!}+\frac{x^8}{8!}+\ldots 
\\ &=\left(1-\frac{x^2}{2!}+\frac{x^4}{4!}-\frac{x^6}{6!}+\frac{x^8}{8!}+\ldots\right)+i\left(x-\frac{x^3}{3!}+\frac{x^5}{5!}-\frac{x^7}{7!}+\ldots\right)
\\ &=\cos{x}+i\sin{x} \end{align*} 
Note: Recall that Euler's formula was useful when solving second order linear homogeneous differential equations with constant coefficients, specifcally for when the characteristic equation had complex roots.}}} \fi


\item The purpose of this problem is to find the Maclaurin series for $\arctan{x}$.  If we attempt to take successive derivatives of $\arctan{x}$ the computation becomes unpleasant rather quickly (try it if you want).  Here is
a simpler alternative. 

\begin{enumerate}

\item Find the Maclaurin series for $\textstyle \frac{1}{1-x}$.

\ifans{\fbox{$1+x+x^2+x^3+x^4+\ldots = \sum_{k=0}^{\infty}{x^k}$.  See \underline{Convergence of Taylor Series} $\#5$. }} \fi

\item Replace $x$ in part (a) with the appropriate quantity to obtain the Maclaurin series for $\textstyle \frac{1}{1+x^2}$.

\ifans{\fbox{\parbox{1\linewidth}{If we replace $x$ in $\textstyle \frac{1}{1-x}$ with $-x^2$ we get $\textstyle \frac{1}{1-(-x^2)}=\frac{1}{1+x^2}$. \\ \\
So the Maclaurin series for $ \frac{1}{1+x^2}$ is $$ 1+(-x^2)+(-x^2)^2+(-x^2)^3+\ldots = 1-x^2+x^4-x^6+\ldots = \sum_{k=0}^{\infty}{(-1)^k x^{2k}}$$  }}}\fi

\item Integrate the answer in part (b) term-by-term to obtain the Maclaurin series for $\arctan{x}$.

\ifans{\fbox{\parbox{1\linewidth}{We know that $\int{\frac{1}{1+x^2}} = \arctan{x}+C$.  \\ 
So $\arctan{x} = \int(1-x^2+x^4-x^6+\ldots) \text{ } dx - C=\left[x-\frac{x^3}{3}+\frac{x^5}{5}-\frac{x^7}{7}+\ldots \right] - C$. \\
Since $\arctan(0)=0$, we have $C=0$, and therefore $\arctan{x}=\sum_{k=0}^{\infty}{\frac{(-1)^k \text{ } x^{2k+1}}{2k+1}}$.   }}} \fi

\end{enumerate}

\item Find the first four nonzero terms of the Maclaurin series for $f(x)=e^{(x^2)}\arctan{x}$ by multiplying the Maclaurin series of the factors.  See the previous problem for the Maclaurin series for $\arctan{x}$.

\ifans{\fbox{\parbox{1\linewidth}{Since $\textstyle e^x=1+x+\frac{x^2}{2!}+\frac{x^3}{3!}+\ldots$ \text{ }, we have $\textstyle e^{(x^2)}=1+x^2+\frac{x^4}{2!}+\frac{x^6}{3!}+\ldots$ 
\\ \\ So \begin{align*}  e^{(x^2)}\arctan{x} &= \left(1+x^2+\frac{x^4}{2!}+\frac{x^6}{3!}+\ldots\right) \left(x-\frac{x^3}{3}+\frac{x^5}{5}-\frac{x^7}{7}+\ldots\right) 
\\ &=(1)(x) +\left[(1)\left(-\frac{x^3}{3}\right)+(x^2)(x)\right]+\left[(1)\left(\frac{x^5}{5}\right)+(x^2)\left(-\frac{x^3}{3}\right)+\left(\frac{x^4}{2!}\right)(x) \right] 
\\ &+ \left[(1)\left(-\frac{x^7}{7}\right)+(x^2)\left(\frac{x^5}{5}\right)+\left(\frac{x^4}{2!}\right)\left(-\frac{x^3}{3!}\right)+\left(\frac{x^6}{3!}\right)(x) \right] +\ldots
\\ &=x+\frac{2}{3}x^3+\frac{11}{30}x^5+\frac{2}{35}x^7+\ldots \end{align*} }}} \fi

\item Consider the function $f(x)=\sin{x}\cos{x}$.

\begin{enumerate}

\item Find the first three nonzero terms of the Maclaurin series for $f(x)$ by multiplying the Maclaurin series of the factors.

\ifans{\fbox{$x-\frac{2}{3}x^3+\frac{2}{15}x^5-\ldots$ }} \fi

\item Confirm your answer in part (a) by using the trigonometric identity \\$\sin{2x}=2\sin{x}\cos{x}$.

\ifans{\fbox{\parbox{1\linewidth}{$\sin{x}\cos{x}=\frac{1}{2}\sin{2x}=\frac{1}{2}\left(2x-\frac{(2x)^3}{3!}+\frac{(2x)^5}{5!}-\ldots \right)=x-\frac{2}{3}x^3+\frac{2}{15}x^5-\ldots$ }}} \fi

\end{enumerate}

\item Find the first three nonzero terms of the Maclaurin series for $f(x)=\tan{x}$ by performing a long division on the Maclaurin series for $\sin{x}$ and $\cos{x}$.

\ifans{\fbox{$x+\frac{1}{3}x^3+\frac{2}{15}x^5+\ldots$; Detailed Solution: \textcolor{blue}{\href{http://www.math.drexel.edu/classes/Calculus/resources/Math123HW/Solutions/123_14_Diff_Int_Series_06.pdf}{Here}}}} \fi

\item Use the result in $\#6$ to find the first three nonzero terms of the Maclaurin series for $f(x)=\sec^2{x}$.

\ifans{\fbox{$\sec^2{x}=\frac{d}{dx}(\tan{x})=\frac{d}{dx}{\left[x+\frac{1}{3}x^3+\frac{2}{15}x^5+\ldots \right]}=1+x^2+\frac{2}{3}x^4+\ldots$ }} \fi

\item Use a Maclaurin series to approximate $\int_0^1{\cos(x^2)} \text{ } dx $ to four decimal-place accuracy.

\ifans{\fbox{\parbox{1\linewidth}{$\cos{x}=\sum_{k=0}^{\infty}{\frac{(-1)^k \text{ } x^{2k}}{(2k)!}}$, so $\cos{(x^2)}=\sum_{k=0}^{\infty}{\frac{(-1)^k \text{ } x^{4k}}{(2k)!}}$.
\\ \\  Thus  $\int_0^1{\cos(x^2)} \text{ } dx = \left.\sum_{k=0}^{\infty}{\frac{(-1)^k \text{ } x^{4k+1}}{(2k)!(4k+1)}}\right|_0^1=\sum_{k=0}^{\infty}{\frac{(-1)^k}{(2k)!(4k+1)}}$, 
\\ which is an alternating series that converges to $S=\int_0^1{\cos(x^2)} \text{ } dx$.
\\ \\ Thus $|S-s_n|<a_{n+1}$, where $s_n$ is the $n$-th partial sum and $\textstyle a_k=\frac{1}{(2k)!(4k+1)}$.
\\ \\ For four decimal-place accuracy, we want 
\begin{align*} a_{n+1} = \frac{1}{(2n+2)!(4n+5)} &\leq 0.00005.
\\ (2n+2)!(4n+5) &\geq 20,000 \end{align*} 
Now for $n=2: (6!)(13) < 20,000$ and for $n=3: (8!)(17) > 20,000$.  
\\ So we want the $3$-rd partial sum $s_3$. 
\\ \\ Therefore  $\int_0^1{\cos(x^2)} \text{ } dx \approx \sum_{k=0}^{3}{\frac{(-1)^k}{(2k)!(4k+1)}} = 1 - \frac{1}{(2!)(5)}+\frac{1}{(4!)(9)}-\frac{1}{(6!)(13)}$.  }}} \fi

\item Use a Maclaurin series to approximate $\int_0^1{\arctan(x^2)} \text{ } dx $ to two decimal-place accuracy.  See problem $\#3$ for the Maclaurin series for $\arctan{x}.$  

\ifans{\fbox{ $\int_0^1{\arctan(x^2)} \text{ } dx \approx \sum_{k=0}^{4}{\frac{(-1)^k}{(2k+1)(4k+3)}} = \frac{1}{3} - \frac{1}{(3)(7)}+\frac{1}{(5)(11)}-\frac{1}{(7)(15)}+\frac{1}{(9)(19)}$. }} \fi

\end{enumerate}

\end{document}