\documentclass[12pt]{article}


\usepackage{amssymb}
\usepackage{amsmath}
\usepackage{fullpage}
%\usepackage{epsfig}
%\usepackage{epstopdf}
\everymath{\displaystyle}
\usepackage{enumerate}
\usepackage{enumitem}
\usepackage[hidelinks]{hyperref}
\usepackage{xcolor}

\newif\ifans

\ansfalse

\begin{document}

\begin{center}
\underline{\LARGE{Polynomial Approximations of Functions}}
\end{center}

\noindent SUGGESTED REFERENCE MATERIAL:

\medskip

\noindent As you work through the problems listed below, you should reference your lecture notes and the relevant chapters in a textbook/online resource.

\medskip

\noindent EXPECTED SKILLS:

\medskip

\begin{itemize}[topsep=0pt]

\item Find and use the local linear and local quadratic approximations of a function $f(x)$ at a specified $x=x_0$.

\item Determine the Maclaurin polynomials of various degrees for a function $f(x)$, and use sigma notation to write the $n$-th Maclaurin polynomial.

\item Determine the Taylor polynomials of various degrees for a function $f(x)$ at a specified $x=x_0$ , and use sigma notation to write the $n$-th Taylor polynomial.

\end{itemize}

\medskip

\noindent PRACTICE PROBLEMS:

\begin{enumerate}

\item Consider the function $f(x)=\sqrt{x}$.

\begin{enumerate}

\item Find the local linear approximation $p_1(x)$ and the local quadratic approximation $p_2(x)$ to $f(x)$ at $x=4$.

\ifans{\fbox{\parbox{1\linewidth}{
$p_1(x)=2+\frac{1}{4}(x-4)$ \\ \\
$p_2(x)=2+\frac{1}{4}(x-4)-\frac{1}{64}(x-4)^2$ \\ \\
Note that $p_1(x)$ and $p_2(x)$ are just the 1st and 2nd Taylor polynomials for $f(x)$ about $x=4$. }}}\fi

\item Approximate $\sqrt{4.1}$ using your answers in part (a).

\ifans{\fbox{\parbox{1\linewidth}{
$p_1(4.1)=2+\frac{1}{4}(4.1-4)=\frac{81}{40}=2.025$ \\ \\
$p_2(4.1)=2+\frac{1}{4}(4.1-4)-\frac{1}{64}(4.1-4)^2=\frac{81}{40}-\frac{1}{6400}=2.02484375$ \\ \\
Calculator: $\sqrt{4.1} \approx 2.024845673$.  }}}\fi

\end{enumerate}

\end{enumerate}

\noindent {\bf For problems 2 -- 4, use the appropriate local linear and local quadratic approximations to approximate the following values.}
\begin{enumerate}
\setcounter{enumi}{1}

\item $\sin{0.1}$

\ifans{\fbox{{\parbox{1\linewidth}{$p_1(x)=p_2(x)=x$, so $\sin{0.1} \approx 0.1$ \\
Calculator: $\sin{0.1} \approx 0.09983341665$ }}}} \fi

\item $\sqrt[3]{28}$

\ifans{\fbox{\parbox{1\linewidth}{
$p_1(x)=3+\frac{1}{27}(x-27)$, so $\sqrt[3]{28} \approx p_1(28) = \frac{82}{27} = 3.\overline{037}$ \\ \\
$p_2(x)=3+\frac{1}{27}(x-27)-\frac{1}{2187}(x-27)^2$, so $\sqrt[3]{28} \approx p_2(28) = \frac{82}{27}-\frac{1}{2187} \approx 3.03657979$  \\ \\
Calculator: $\sqrt[3]{28} \approx 3.036588972$. }}}\fi

\item $\tan{44^\circ}$

\ifans{\fbox{\parbox{1\linewidth}{
$p_1(x)=1+2\left(x-\frac{\pi}{4}\right)$, so $\tan{44^\circ} \approx p_1\left(\frac{\pi}{4}-\frac{\pi}{180}\right) = 1-\frac{\pi}{90} \approx 0.965093415$ \\ \\
$p_2(x)=1+2\left(x-\frac{\pi}{4}\right)+2\left(x-\frac{\pi}{4}\right)^2$, so \\ 
$\tan{44^\circ} \approx p_2\left(\frac{\pi}{4}-\frac{\pi}{180}\right) = 1-\frac{\pi}{90} + \frac{2\pi^2}{(180)^2} \approx 0.9657026498$   \\ \\
Calculator: $\tan{44^\circ} \approx 0.9656887747$. }}}\fi

\item Suppose that the values of $f(x)$ and its first four derivatives at $x=0$ are as follows: \newline \newline
$f(0)=5 \hspace{5 mm} f^{\prime}(0)=-2 \hspace{5 mm} f^{\prime\prime}(0)=0 \hspace{5 mm} f^{\prime\prime\prime}(0)=-1 \hspace{5 mm} f^{(4)}(0)=12$ \newline \newline
Based on this information, list out as many Maclaurin polynomials for $f(x)$ as possible.

\ifans{\fbox{\parbox{1\linewidth}{
$p_0(x)=5$ \\ \\ 
$p_1(x)=p_2(x)=5-2x$ \\ \\
$p_3(x)=5-2x-\frac{1}{6}x^3$ \\ \\
$p_4(x)=5-2x-\frac{1}{6}x^3+\frac{1}{2}x^4$ }}}\fi

\item Find the 4th Maclaurin polynomial $p_4(x)$ for the function $f(x)=2x^4-x^3+6$.

\ifans{\fbox{$p_4(x)=2x^4-x^3+6$.  Why does it make sense that $p_4(x)=f(x)$?}}\fi


\end{enumerate}

\noindent {\bf For problem 7, find the Macluarin polynomials $p_0(x), p_1(x), p_2(x), p_3(x)$, and $p_4(x)$.  Then write the $n$-th Maclaurin polynomial $p_n(x)$ using sigma notation.}
\begin{enumerate}
\setcounter{enumi}{6}

\item $f(x)=\ln(1+x)$

\ifans{\fbox{\parbox{1\linewidth}{$p_0(x)=0 \\ \\ p_1(x)=x \\ \\ p_2(x)=x-\frac{1}{2}x^2 \\ \\ p_3(x)=x-\frac{1}{2}x^2+\frac{1}{3}x^3 
\\ \\ p_4(x)=x-\frac{1}{2}x^2+\frac{1}{3}x^3-\frac{1}{4}x^4 
\\ \\ p_n(x)=\sum_{k=1}^{n}{(-1)^{k+1} \text{ } \frac{x^k}{k}}$ [Does this series look familiar?  Try plugging in $x=1$.] }}} \fi

\end{enumerate}

\noindent {\bf For problems 8 \& 9, find the Taylor polynomials $p_0(x), p_1(x), p_2(x), p_3(x)$, and $p_4(x)$ about $x=x_0$.  Then write the $n$-th Taylor polynomial $p_n(x)$ at $x=x_0$ using sigma notation.}
\begin{enumerate}
\setcounter{enumi}{7}

\item $f(x)=\frac{1}{1-x}; \text{   } x_0=2$

\ifans{\fbox{\parbox{1\linewidth}{$p_0(x)=-1 \\ \\ p_1(x)=-1+(x-2) \\ \\ p_2(x)=-1+(x-2)-(x-2)^2 \\ \\ p_3(x)=-1+(x-2)-(x-2)^2+(x-2)^3 
\\ \\ p_4(x)=-1+(x-2)-(x-2)^2 +(x-2)^3-(x-2)^4
\\ \\ p_n(x)=\sum_{k=0}^{n}{(-1)^{k+1} \text{ } (x-2)^k}$  }}} \fi

\item $f(x)=e^{2x}; \text{   } x_0=\ln{3}$

\ifans{\fbox{\parbox{1\linewidth}{$p_0(x)=9 \\ \\ p_1(x)=9+18(x-\ln{3}) \\ \\ p_2(x)=9+18(x-\ln{3})+18(x-\ln{3})^2 \\ \\ p_3(x)=9+18(x-\ln{3})+18(x-\ln{3})^2+12(x-\ln{3})^3
\\ \\ p_4(x)=9+18(x-\ln{3})+18(x-\ln{3})^2+12(x-\ln{3})^3+6(x-\ln{3})^4
\\ \\ p_n(x)=\sum_{k=0}^{n}{\frac{2^k(9)}{k!} (x-\ln{3})^k}$  }}} \fi

\end{enumerate}



\end{document}