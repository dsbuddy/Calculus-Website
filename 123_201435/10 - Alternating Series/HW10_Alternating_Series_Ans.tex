\documentclass[12pt]{article}


\usepackage{amssymb}
\usepackage{amsmath}
\usepackage{fullpage}
%\usepackage{epsfig}
%\usepackage{epstopdf}
\everymath{\displaystyle}
\usepackage{enumerate}
\usepackage{enumitem}
\usepackage[hidelinks]{hyperref}
\usepackage{xcolor}

\newif\ifans

\anstrue

\begin{document}

\begin{center}
\underline{\LARGE{Alternating Series; Absolute/Conditional Convergence}}
\end{center}

\noindent SUGGESTED REFERENCE MATERIAL:

\medskip

\noindent As you work through the problems listed below, you should reference your lecture notes and the relevant chapters in a textbook/online resource.

\bigskip

\noindent EXPECTED SKILLS:

\medskip

\begin{itemize}[topsep=0pt]

\item Determine if an alternating series converges using the Alternating Series Test.

\item Analyze the absolute values of the terms of a series and determine if it converges.

\item Use any of the previously discussed convergence tests to determine if a series with negative terms converges absolutely, converges conditionally, or diverges.

\item Find a partial sum that approximates a convergent alternating series to some specified accuracy.

\end{itemize}

\medskip

\noindent PRACTICE PROBLEMS:

\medskip

\noindent {\bf For problems 1 -- 3, show that the series converges by verifying that it satisfies the hypotheses of the Alternating Series Test, or show that the series does not satisfy the hypotheses of the Alternating Series Test.  If the latter, then use some other test to determine if the series converges or diverges.}

\medskip

\begin{enumerate}

\item $\frac{1}{2}-\frac{1}{4}+\frac{1}{8}-\frac{1}{16}+\frac{1}{32}-\frac{1}{64}+\ldots$

\ifans{\fbox{\parbox{1\linewidth}{This is the series $\sum_{k=0}^{\infty}{(-1)^k \left(\frac{1}{2}\right)^k}$.   
\newline  \newline  We have $\textstyle a_k=\left(\frac{1}{2}\right)^k$, so $\frac{a_{k+1}}{a_k}=\frac{1}{2}<1$ for $k \geq 0$.     
\newline \newline Thus the sequence $\left\{a_k \right\}_{k=0}^{+\infty}$ is decreasing; also, $\lim_{k \rightarrow \infty}{a_k}=0$.
\newline \newline So by the Alternating Series Test the series converges.
\newline  [Of course this is just a geometric series with $\textstyle r=-\frac{1}{2}$.]
  }}}\fi

\item $\sum_{k=2}^{\infty}{(-1)^k \text{ } \frac{k^2}{2+k^3}}$

\ifans{\fbox{\parbox{1\linewidth}{We have $a_k=\frac{k^2}{2+k^3}$, so $a_k^\prime=-\frac{k (k^3-4)}{(2+k^3)^2}<0$ for $k\geq2$.     
\newline \newline Thus the sequence $\left\{a_k \right\}_{k=2}^{+\infty}$ is decreasing; also, $\lim_{k \rightarrow \infty}{a_k}=0$.
\newline \newline So by the Alternating Series Test the series converges.
  }}}\fi

\item $\sum_{k=1}^{\infty}{(-1)^{k+1} \text{ } \frac{1+2k}{3k}}$

\ifans{\fbox{\parbox{1\linewidth}{$\lim_{k \rightarrow \infty}{\frac{1+2k}{3k}}=\frac{2}{3}\neq0$, so the hypotheses of the Alternating Series Test are not satisfied.
\newline \newline Now $\lim_{k \rightarrow \infty}{\left((-1)^{k+1} \text{ } \frac{1+2k}{3k}\right)}$ DNE (half of the terms are approaching $\tfrac{2}{3}$, the other half are approaching $-\tfrac{2}{3}$).  Since the limit is not zero, the series diverges by the Divergence Test.  }}}\fi

\end{enumerate}

\noindent {\bf For problems 4 -- 13, determine if the series converges absolutely, converges conditionally, or diverges.}

\medskip

\begin{enumerate}
\setcounter{enumi}{3}

\item $\frac{1}{2}+\frac{1}{4}+\frac{1}{8}-\frac{1}{16}-\frac{1}{32}-\frac{1}{64}+\frac{1}{128}+\frac{1}{256}+\frac{1}{512}-\frac{1}{1024}-\ldots$

\ifans{\fbox{converges absolutely}} \fi

\item $\sum_{k=1}^{\infty}{(-1)^{k+1} \text{ } \frac{k^2}{2+k^3}}$

\ifans{\fbox{converges conditionally}} \fi

\item $\sum_{k=1}^{\infty}{(-1)^{k} \text{ } \frac{1}{4k^2+9}}$

\ifans{\fbox{converges absolutely}} \fi

\item $\sum_{k=1}^{\infty}{(-1)^{k} \arcsin{\left(\frac{1}{k}\right)}}$

\ifans{\fbox{converges conditionally; Detailed Solution: \textcolor{blue}{\href{http://www.math.drexel.edu/classes/Calculus/resources/Math123HW/Solutions/123_10_Alternating_Series_7.pdf}{Here}}}} \fi

\item $\sum_{k=1}^{\infty}{\frac{(-1)^{k+1}}{k!}}$

\ifans{\fbox{converges absolutely}} \fi

\item $\sum_{k=1}^{\infty}{\cos({k\pi})ke^{-k}}$

\ifans{\fbox{converges absolutely; Detailed Solution: \textcolor{blue}{\href{http://www.math.drexel.edu/classes/Calculus/resources/Math123HW/Solutions/123_10_Alternating_Series_9.pdf}{Here}}}} \fi

\item $\sum_{k=1}^{\infty}{(-1)^k e^{(1/k) } }$

\ifans{\fbox{diverges}} \fi

\item $\sum_{k=3}^{\infty}{(-1)^k \frac{\sqrt{k}}{k-2} }$

\ifans{\fbox{converges conditionally}} \fi

\item $\sum_{k=1}^{\infty}{\frac{\sin{k}}{k^4+4^k} }$

\ifans{\fbox{converges absolutely}} \fi

\item $\sum_{k=1}^{\infty}{(-1)^k \frac{(2k+1)!}{200^k} }$

\ifans{\fbox{diverges}} \fi

\item $\sum_{k=3}^{\infty}{\sin\left[(2k+1)\frac{\pi}{2}\right] \text{ } \frac{\ln{k}}{k} }$

\ifans{\fbox{converges conditionally}} \fi

\end{enumerate}

\noindent {\bf For problems 15 -- 17, the series converge to some sum $S$.  Find the smallest value of $n$ so that the $n$-th partial sum $s_n$ will guarantee the approximation of $S$ to the required accuracy.}

\begin{enumerate}
\setcounter{enumi}{14}

\item $\sum_{k=1}^{\infty}{\frac{(-1)^{k+1}}{k!}}; \text{         }|S-s_n|<0.001$

\ifans{\fbox{$n = 6$; Detailed Solution: \textcolor{blue}{\href{http://www.math.drexel.edu/classes/Calculus/resources/Math123HW/Solutions/123_10_Alternating_Series_15.pdf}{Here}}}} \fi

\item $\sum_{k=1}^{\infty}{\frac{(-1)^k}{\sqrt[4]{k+7}}}; \text{         }|S-s_n|<0.2$

\ifans{\fbox{$n = 617$}} \fi

\item $\sum_{k=1}^{\infty}{\frac{(-1)^{k+1}}{2^k k}}; \text{         }|S-s_n|<0.0005$

\ifans{\fbox{$n = 7$}} \fi

\item Is the $300^{\text {th}}$ partial sum of $\textstyle \sum_{k=1}^{\infty}{\frac{(-1)^{k+1}}{2^k k}}$ an overestimate or an underestimate of the sum of the series?  Justify your answer.

\ifans{\fbox{\parbox{1\linewidth}{Let $S$ be the sum of the series.  Now the first term of the series is positive, the second is negative, the third positive, the fourth negative, and so on.  
Thus the sequence of odd partial sums is decreasing (and converges to $S$) while the sequence of even partial sums is increasing (and converges to $S$).  So every odd partial sum is an
overestimate of $S$ and every even partial sum is an underestimate of $S$.  Therefore the $300^{\text {th}}$ partial sum of the series is an underestimate of $S$.       }}} \fi




\end{enumerate}

\end{document}