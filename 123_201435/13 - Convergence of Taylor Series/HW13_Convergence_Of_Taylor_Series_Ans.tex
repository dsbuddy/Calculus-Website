\documentclass[12pt]{article}


\usepackage{amssymb}
\usepackage{amsmath}
\usepackage{fullpage}
%\usepackage{epsfig}
%\usepackage{epstopdf}
\everymath{\displaystyle}
\usepackage{enumerate}
\usepackage{enumitem}
\usepackage[hidelinks]{hyperref}
\usepackage{xcolor}


\newif\ifans

\anstrue

\begin{document}

\begin{center}
\underline{\LARGE{Convergence of Taylor Series}}
\end{center}

\noindent SUGGESTED REFERENCE MATERIAL:

\medskip

\noindent As you work through the problems listed below, you should reference your lecture notes and the relevant chapters in a textbook/online resource.

\medskip

\noindent EXPECTED SKILLS:

\medskip

\begin{itemize}[topsep=0pt]

\item Know (i.e. memorize) the Remainder Estimation Theorem, and use it to find an upper bound on the error in approximating a function with its Taylor polynomial.

\item Find the value(s) of $x$ for which a Taylor series converges to a function $f(x)$.

\end{itemize}

\medskip

\noindent PRACTICE PROBLEMS:

\medskip

\begin{enumerate}

\item Find an upper bound for the remainder error if the $4$th Maclaurin polynomial for $f(x)=\cos{x}$ is used to approximate $\cos{5^\circ}$.



\ifans{\fbox{\parbox{1\linewidth}{If $f(x)=\cos{x}$, then $\textstyle|f^{(5)}(x)| \leq 1$ for all $x$, and so by the Remainder Estimation Theorem, $\textstyle |R_4(\frac{\pi}{36})| \leq \frac{1}{5!} \left(\frac{\pi}{36} - 0 \right)^5$.
\\ \\ This can be verified with a calculator as follows: \\ The $4$th Maclaurin polynomial for $\cos{x}$ is $\textstyle p_4(x)=1-\frac{1}{2!}x^2+\frac{1}{4!}x^4$.  
\\  Thus $\textstyle \cos{\left(\frac{\pi}{36}\right)} \approx p_4\left(\frac{\pi}{36}\right) \approx 0.996194698705$
\\ Now a calculator tells us that $\textstyle \cos{\left(\frac{\pi}{36}\right)} \approx 0.996194698092$.
\\ So $\textstyle |R_4(\frac{\pi}{36})| = |\cos({\frac{\pi}{36}}) - p_4(\frac{\pi}{36})| \approx 6 \times 10^{-10} < \frac{1}{5!} \left(\frac{\pi}{36}\right)^5 \approx 4.218 \times 10^{-8}$. 
   }}} \fi

\item Find an upper bound for the remainder error if the $2$nd Maclaurin polynomial for $f(x)=e^x$ is used to approximate $\sqrt{e}$?  
\\ Note: You may assume that $\sqrt{e} < 2$ (this should be clear since $\sqrt{e}<\sqrt{3}<\sqrt{4}=2$). 

\ifans{\fbox{\parbox{1\linewidth}{Note that $f(x)=e^x$ is an increasing function.  So for all $x$ on the interval $\textstyle [0, \frac{1}{2}]$  we have $|f^{(3)}(x)| = e^x \leq e^{\frac{1}{2}} < 2$
, and so by the Remainder Estimation Theorem, $\textstyle |R_2(\frac{1}{2})| \leq \frac{2}{3!} \left(\frac{1}{2} - 0 \right)^3 = \frac{1}{24}$.
\\ \\ This can be verified with a calculator as follows: \\ The $2$nd Maclaurin polynomial for $e^x$ is $\textstyle p_2(x)=1+x+\frac{1}{2}x^2$.  
\\  Thus $\textstyle \sqrt{e} \approx p_2(\frac{1}{2}) = 1.625$
\\ Now a calculator tells us that $\textstyle \sqrt{e} \approx 1.648721271$.
\\ So $\textstyle |R_2(\frac{1}{2})| = |\sqrt{e} - p_2(\frac{1}{2})| \approx 0.023721271 < \frac{1}{24} = 0.041\overline{6}$. 
   }}} \fi

\item Find the smallest value of $n$ that is needed so that the $n$-th Macluarin polynomial $p_n(x)$ approximates $\sqrt{e}$ to four decimal-place accuracy.  In other words, find the smallest value of $n$
so that the $n$-th remainder $\textstyle |R_n(\frac{1}{2})| \leq 0.00005$.
\\ Note: You may assume that $\sqrt{e} < 2$ (this should be clear since $\sqrt{e}<\sqrt{3}<\sqrt{4}=2$). 

\ifans{\fbox{\parbox{1\linewidth}{Note that $f(x)=e^x$ is an increasing function and $f^{(n)}(x)=e^x$ for all $n$.  So for all $x$ on the interval $\textstyle [0, \frac{1}{2}]$  we have $|f^{(n+1)}(x)| = e^x \leq e^{\frac{1}{2}} < 2$
, and so by the Remainder Estimation Theorem, $\textstyle |R_n(\frac{1}{2})| \leq \frac{2}{(n+1)!} \left(\frac{1}{2} - 0 \right)^{n+1} = \frac{1}{2^n (n+1)!}$.
\\ \\ So we want $\textstyle \frac{1}{2^n (n+1)!} \leq 0.00005$, or $2^n (n+1)! \geq 20,000$.
\\ For $n=4: 2^4(5!) = 16(120) < 20,000$.
\\ For $n=5: 2^5(6!) = 32(720) > 20,000$.
\\ So we should let $n=5$, i.e. the $5$-th Macluarin polynomial for $e^{x}$ approximates $\sqrt{e}$ to four decimal-place accuracy. }}} \fi

\item Find the smallest value of $n$ so that the Taylor polynomial for $f(x)=\ln(x)$ about $x_0=1$ approximates $\ln(1.2)$ to three decimal-place accuracy.

\ifans{\fbox{$n=3$.; Detailed Solution: \textcolor{blue}{\href{http://www.math.drexel.edu/classes/Calculus/resources/Math123HW/Solutions/123_13_Convergence_Of_Taylor_Series_04.pdf}{Here}}}} \fi

\item The purpose of this problem is to show that the Maclaurin series for $f(x)=\cos{x}$ converges to $\cos{x}$ for all $x$.

\begin{enumerate}

\item Find the Maclaurin series for $f(x)=\cos{x}$.

\ifans{\fbox{$\sum_{k=0}^{\infty}{\frac{(-1)^k x^{2k}}{(2k)!}}$. }} \fi

\item Find the interval of convergence for this Maclaurin series.

\ifans{\fbox{$(-\infty, +\infty)$.  See \underline{Power Series} \#15.}} \fi

\item Show that the $n$-th remainder goes to $0$ as $n$ goes to $+\infty$, \\ i.e. show that $\lim_{n \to +\infty}{|R_n(x)|=0}$.

\ifans{\fbox{\parbox{1\linewidth}{If $f(x)=\cos(x)$, then $|f^{(n+1)}(x)| \leq 1$ for all $n$ and for all $x$. 
\\ \\ So by the Remainder Estimation Theorem, $\textstyle 0 \leq |R_n(x)| \leq \frac{1}{(n+1)!} |x|^{n+1}$.
\\ \\ Now $\lim_{n \to +\infty}{0} = \lim_{n \to +\infty}{ \frac{1}{(n+1)!} |x|^{n+1}} = 0$.
\\ \\ So by the Squeeze Theorem $\lim_{n \to +\infty}{|R_n(x)|=0}$      }}} \fi

\end{enumerate}

\item Show that the Maclaurin series for $\textstyle f(x)=\frac{1}{1-x}$ converges to $f(x)$ for all $x$ in its interval of convergence.

\ifans{\fbox{\parbox{1\linewidth}{The Maclaurin series for $\textstyle f(x)=\frac{1}{1-x}$ is $\textstyle 1+x+x^2+x^3+x^4+\ldots = \sum_{k=0}^{\infty}{x^k}$, which is a geometric series with
$a=1$ and $r=x$.  Thus the series converges if, and only if, $-1 < x < 1$.  For these values of $x$, the series converges to $\textstyle \frac{a}{1-r}=\frac{1}{1-x}=f(x)$. }}} \fi

\item The pupose of this problem is to show that it is possible for a function $f(x)$ to have a Maclaurin series that converges for all $x$ but does not always converge to $f(x)$.
\\ \\ Consider the piecewise function $f(x) = \begin{cases} 
e^{(-1/x^2)}, & \text{if } x \neq 0 \\
0, & \text{if } x = 0 \end{cases}$ .

\begin{enumerate}

\item Use the definition of the derivative $f^{\prime}(x)=\lim_{h \to 0}{\tfrac{f(x+h)-f(x)}{h}}$ to show that $f'(0)=0$.  Hint:  Make the substitution $\textstyle t=\frac{1}{h}$ and compute the one-sided limits as $h \to 0^+$ and $h \to 0^-$.

\ifans{\fbox{\parbox{1\linewidth}{$f^{\prime}(0) = \lim_{h \to 0}{\frac{f(0+h)-f(0)}{h}} = \lim_{h \to 0}{\frac{e^{(-1/h^2)}-0}{h}}$
\\ \\ Now let $\textstyle t=\frac{1}{h}$ and examine the limit as $h \to 0^+$ and $h \to 0^-$.
\\ \\ So \text{ } $\lim_{h \to 0^+}{\frac{e^{(-1/h^2)}}{h}}=\lim_{t \to +\infty}{\frac{e^{-t^2}}{\frac{1}{t}}}=\lim_{t \to +\infty}{\frac{t}{e^{t^2}}}=\lim_{t \to +\infty}{\frac{1}{2te^{t^2}}}=0$, 
\\ \\ and $\lim_{h \to 0^-}{\frac{e^{(-1/h^2)}}{h}}=\lim_{t \to -\infty}{\frac{e^{-t^2}}{\frac{1}{t}}}=\lim_{t \to -\infty}{\frac{t}{e^{t^2}}}=\lim_{t \to -\infty}{\frac{1}{2te^{t^2}}}=0$.
\\ \\ Therefore $f^{\prime}(0)=0$. }}} \fi

\item Assuming that $f^{(n)}(0)=0$ for $n \geq 2$, find the Macluarin series for $f(x)$ and the interval of convergence for the series.

\ifans{\fbox{\parbox{1\linewidth}{Since $f(0)=0$ from the function definiton, $f^{\prime}(0)=0$ by part (a), and $f^{(n)}(0)=0$ for $n \geq 2$ by assumption, the Macluarin series for $f(x)$ is $$0 + 0x + 
\frac{0x^2}{2!}+\frac{0x^3}{3!}+\ldots = \sum_{k=0}^{\infty}{0}=0.$$ \\Thus the series converges (to $0$) for all $x$, i.e. the interval of convergance is $(-\infty, +\infty)$.  }}} \fi

\item Find the values of $x$ for which the Maclaurin series converges to $f(x)$.

\ifans{\fbox{\parbox{1\linewidth}{From part (b) we know that the Maclaurin series for $f(x)$ converges to $0$ for all $x$.  Now $f(0)=0$, but if $x \neq 0$ then $f(x)=e^{(-1/x^2)} > 0$.  Thus the 
Maclaurin series for $f(x)$ converges for all $x$ but only converges to $f(x)$ for $x=0$. }}} \fi

\end{enumerate} 




\end{enumerate}

\end{document}