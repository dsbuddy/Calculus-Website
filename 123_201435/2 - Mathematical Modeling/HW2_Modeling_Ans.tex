\documentclass[12pt]{article}


\usepackage{amssymb}
\usepackage{amsmath}
\usepackage{fullpage}
\usepackage{epsfig}
\usepackage{epstopdf}
\everymath{\displaystyle}
\usepackage{enumerate}
\usepackage{enumitem}
\usepackage[hidelinks]{hyperref}
\usepackage{xcolor}


\newif\ifans

\anstrue

\begin{document}

\begin{center}
\underline{\LARGE{Mathematical Modeling}}
\end{center}

\noindent SUGGESTED REFERENCE MATERIAL:

\medskip

\noindent As you work through the problems listed below, you should reference your lecture 
notes and the relevant chapters in a textbook/online resource.

\bigskip

\noindent EXPECTED SKILLS:

\begin{itemize}

\item Set up (and possibly solve) an initial-value problem based on a description of some phenomenon.

\item Know (i.e. memorize) the initial-value problems and solutions for both the exponential growth and exponential decay models, and be able to answer questions based on these models. 

\item You do not need to memorize equations for other models (e.g. Logistic Model, Newton’s Law of Cooling), but be able to answer questions based on these models if you are given the equations.

\end{itemize}

\noindent PRACTICE PROBLEMS:

\begin{enumerate}

\item Suppose that $y=y(t)$ is a function which outputs the amount of a quantity at time $t$. For each of the following, set up an initial value problem whose solution is $y(t)$.

\begin{enumerate}

\item The quantity $y=y(t)$ increases at a rate that is proportional to the cube of the amount present.  And, at $t=0$, the amount present is $y_0$.  

\ifans{\fbox{$\left\{\begin{array}{l}
\frac{dy}{dt}=ky^3\\
\\
y(0)=y_0
\end{array}\right.$, where $k$ is a positive constant.}} \fi

\item The quantity $y=y(t)$ decreases at a rate that is one-half of the square root of the amount present.  And, at $t=0$, the amount present is $y_0$.

\ifans{\fbox{$\left\{\begin{array}{l}
\frac{dy}{dt}=-\frac{1}{2}\sqrt{y}\\
\\
y(0)=y_0
\end{array}\right.$}} \fi

\end{enumerate}

\item Suppose that a quantity $y=y(t)$ changes in such a way that $y^{\prime}=k\sqrt[5]{y}$, where $k>0$.  Describe in words how $y$ changes.

\ifans{\fbox{The quantity increases at a rate proportional to the fifth-root of the amount present.}} \fi

\item Suppose that a ball is moving along a straight line through a resistive medium in such a way that its velocity $v=v(t)$ decreases at a rate that is twice the square root of the velocity.  Suppose that at time $t=3$ seconds, the velocity of the ball is 16 meters/second.

\begin{enumerate}

\item Set up and solve an initial value problem whose solution is $v=v(t)$.  Express your solution as an explicit function of $t$.

\ifans{\fbox{\parbox{1\linewidth}{Initial Value Problem:
$\left\{\begin{array}{l}
\frac{dv}{dt}=-2\sqrt{v}\\
\\
v(3)=16
\end{array}\right.$; Solution:
$v(t)=(7-t)^2$}}} \fi

\item At what time does the ball come to a complete stop?

\ifans{\fbox{The ball will stop at $t=7$ seconds.}} \fi

\end{enumerate}

\item In a certain nutrient culture, a cell of the bacterium E. coli divides into two every 10 minutes.  Let $y=y(t)$ be the number of cells that are present $t$ minutes after 6 cells are placed in the culture.  Assume that the growth of the bacteria obeys the exponential growth model.

\begin{enumerate}

\item Find a formula for $y(t)$

\ifans{\fbox{$y(t)=6(2)^{t/10}$}} \fi

\item How long will it take for the bacterial population to reach 600 cells?

\ifans{\fbox{$t=\frac{10\ln{(100)}}{\ln{2}}=10\log_2 100 \approx 66.64$ minutes}} \fi

\end{enumerate}

\item The population of the United States was 3.9 million in 1790 and 178 million in 1960.  

\begin{enumerate}

\item Assuming exponential growth, find a formula for $P(t)$, the population of the United States $t$ years after 1790.

\ifans{\fbox{$P(t)=3.9\left(\frac{1780}{39}\right)^{t/170}$}} \fi

\item Using the model from part (a), estimate the population in the year 2000.

\ifans{\fbox{$P(210) = 3.9 \left( \frac{1780}{39} \right)^{210/170} \approx437.38$ million people}} \fi

\end{enumerate}

\item Carbon 14, an isotope of Carbon, is radioactive and decays at a rate proportional to the amount present.  Its half life is 5730 years.  If there were 20 grams of Carbon 14 present originally, how much would be left after 2000 years?

\ifans{\fbox{$y(t)=20(2)^{-t/5730}$.  Thus, $y(2000)=20(2)^{-2000/5730}\approx 15.702$ grams.}} \fi

\item A scientist wants to determine the half life of a certain radioactice substance.  She determines that in exactly 5 days an 80 milligram sample decays to 10 milligrams.  Based on this data, what is the half life?

\ifans{\fbox{$\frac{5}{3}$ days; Detailed Solution: \textcolor{blue}{\href{http://www.math.drexel.edu/classes/Calculus/resources/Math123HW/Solutions/123_02_Modeling_07.pdf}{Here}}}} \fi 

\item An unknown amount of a radioactive substance is being studied.  After two days, the mass is 15 grams.  After eight days, the mass is 9 grams.  Assume exponential decay.

\begin{enumerate}

\item How much of the substance was there initially?

\ifans{\fbox{$y_0 = 9\left(\frac{3}{5}\right)^{-4/3}\approx 17.78$ grams}} \fi

\item What is the half-life of the substance?

\ifans{\fbox{$T_{1/2}=-\frac{6\ln{2}}{\ln{\left(\frac{9}{15}\right)}}\approx 8.14$ days}} \fi

\end{enumerate}

\item The differential equation $\frac{dy}{dt}=k\left(1-\frac{y}{L}\right)y$ for $k>0$ models the change in the size $y=y(t)$ of a population inhabiting an ecological system with a carrying capacity of $L$ individuals.

\begin{enumerate}

\item What are the constant solutions to this differential equation?  Explain the meaning of these constant solutions in terms of the model.

\ifans{\fbox{\parbox{1\linewidth}{The constant solutions correspond to $\frac{dy}{dt}=0$.  This occurs for $y=0$ and $y=L$.  If $y=0$, the population is at size 0.  And, if $y=L$, the population is at the system's carrying capacity.  Either way, there is no growth.}}} \fi

\item For what size of the population will the population be growing the fastest?

\ifans{\fbox{The population is growing fastest when $y=\frac{L}{2}$; Detailed Solution: \textcolor{blue}{\href{http://www.math.drexel.edu/classes/Calculus/resources/Math123HW/Solutions/123_02_Modeling_09.pdf}{Here}}}} \fi

\end{enumerate}

\item Newton's Law of Cooling states that the rate at which an object cools (or warms) is proportional to the difference in temperature between the object and the surrounding environment.  Suppose an object with initial temperature $T_0$ is placed in a room with a constant temperature of $T_e$.  If $T(t)$ represents the temperature of the object at time $t$, then Newton's Law of Cooling can be expressed with the following initial value problem:

$$\left\{\begin{array}{l}
\frac{dT}{dt}=k(T_e-T), k>0\\
\\
T(0)=T_0
\end{array}\right.$$

\begin{enumerate}

\item Use separation of variables to show that $T(t)=T_e + (T_0 - T_e)e^{-kt}$ is the particular solution to this initial value problem.

\item An object taken from an oven has a temperature of $350^{\circ}$ and is left to cool in a room that is $65^{\circ}$.  After 1 hour, the temperature drops to $200^{\circ}$.  Give a formula for the function $T(t)$, the temperature at time $t$.

\ifans{\fbox{$T(t)=285\left(\frac{9}{19}\right)^t+65$}} \fi

\item For the object described in part (b), evaluate $\lim_{t \rightarrow \infty}{T(t)}$ and interpret your answer.

\ifans{\fbox{$\lim_{t \rightarrow \infty}{T(t)}=65$.  So over time the object will approach the temperature of the room.}} \fi

\end{enumerate}

\end{enumerate}

\end{document}