\documentclass[12pt]{article}


\usepackage{amssymb}
\usepackage{amsmath}
\usepackage{fullpage}
%\usepackage{epsfig}
%\usepackage{epstopdf}
\everymath{\displaystyle}
\usepackage{enumerate}
\usepackage{enumitem}
\usepackage[hidelinks]{hyperref}
\usepackage{xcolor}

\newif\ifans

\ansfalse

\begin{document}

\begin{center}
\underline{\LARGE{Convergence Tests: Divergence, Integral, and p-Series Tests}}
\end{center}

\noindent SUGGESTED REFERENCE MATERIAL:

\medskip

\noindent As you work through the problems listed below, you should reference your lecture notes and the relevant chapters in a textbook/online resource.

\bigskip

\noindent EXPECTED SKILLS:

\medskip

\begin{itemize}[topsep=0pt]

\item Recognize series that cannot converge by applying the Divergence Test.

\item Use the Integral Test on appropriate series (all terms positive, corresponding function is decreasing and continuous) to make a conclusion about the convergence of the series.

\item Recognize a $p$-series and use the value of $p$ to make a conclusion about the convergence of the series.

\item Use the algebraic properties of series.

\end{itemize}

\bigskip

\noindent PRACTICE PROBLEMS:

\medskip

\noindent {\bf For problems 1 -- 9, apply the Divergence Test.  What, if any, conlcusions can you draw about the series?}

\begin{enumerate}

\item $\sum_{k=1}^{\infty}(-1)^k$

\ifans{\fbox{\parbox{1\linewidth}{$\lim_{k \rightarrow \infty}(-1)^k$ DNE, so by the Divergence Test the series diverges. \\ \\  Also, recall that this series is a geometric series with ratio $r=-1$, which  \\ confirms that it must diverge. }}} \fi

\item $\sum_{k=1}^{\infty}(-1)^k \text{ } \frac{1}{k}$

\ifans{\fbox{$\lim_{k \rightarrow \infty}(-1)^k \text{ } \frac{1}{k}=0$, so the Divergence Test is inconclusive. }} \fi

\item $\sum_{k=3}^{\infty}\frac{\ln{k}}{k}$

\ifans{\fbox{$\lim_{k \rightarrow \infty}\frac{\ln{k}}{k}=0$, so the Divergence Test is inconclusive. }} \fi

\item $\sum_{k=1}^{\infty}\frac{\ln{6k}}{\ln{2k}}$

\ifans{\fbox{$\lim_{k \rightarrow \infty}\frac{\ln{6k}}{\ln{2k}}=1\neq0$ [see \underline{Sequences} problem \#26], so by the Divergence Test the series diverges.}} \fi

\item $\sum_{k=1}^{\infty}{ke^{-k}}$

\ifans{\fbox{\parbox{1\linewidth}{$\lim_{k \rightarrow \infty}{ke^{-k}}=0$ [see \underline{Limits at Infinity Review} problem \#6], so the Divergence Test is inconclusive.}}} \fi

\item $\sum_{k=1}^{\infty}\frac{e^k-e^{-k}}{e^k+e^{-k}}$

\ifans{\fbox{\parbox{1\linewidth}{$\lim_{k \rightarrow \infty}\frac{e^k-e^{-k}}{e^k+e^{-k}}=1\neq0$ [see \underline{Sequences} problem \#21], \\ \\ so by the Divergence Test the series diverges.}}} \fi

\item $\sum_{k=1}^{\infty}\left(1+\frac{1}{k}\right)^k$

\ifans{\fbox{\parbox{1\linewidth}{$\lim_{k \rightarrow \infty}\left(1+\frac{1}{k}\right)^k=e\neq0$ [see \underline{Sequences} problem \#34], \\ \\ so by the Divergence Test the series diverges.}}} \fi

\item $\sum_{k=1}^{\infty}(\sqrt{k^2+8k-5}-k)$

\ifans{\fbox{\parbox{1\linewidth}{$\lim_{k \rightarrow \infty}(\sqrt{k^2+8k-5}-k)=4\neq0$ [see \underline{Sequences} problem \#28], \\ \\ so by the Divergence Test the series diverges.}}} \fi

\item $\sum_{k=2}^{\infty}(\sqrt{k^2+3}-\sqrt{k^2-4})$

\ifans{\fbox{$\lim_{k \rightarrow \infty}(\sqrt{k^2+3}-\sqrt{k^2-4})=0$, so the Divergence Test is inconclusive.; Detailed Solution: \textcolor{blue}{\href{http://www.math.drexel.edu/classes/Calculus/resources/Math123HW/Solutions/123_08_Convergence_Tests_09.pdf}{Here}}}} \fi


\end{enumerate}

\noindent{\bf For problems 10 -- 20, determine if the series converges or diverges by applying the Divergence Test, Integral Test, or noting that the series is a $p$-series.  Explicitly state what test you are using.  If you use the Integral Test, you must first verify that the test is applicable.  If the series is a $p$-series, state the value of $p$.}

\begin{enumerate}
\setcounter{enumi}{9}

\item $\sum_{k=3}^{\infty}\frac{\ln{k}}{k}$

\ifans{\fbox{The series diverges by the Integral Test.}} \fi

\item $\sum_{k=1}^{\infty}{ke^{-k}}$

\ifans{\fbox{The series converges by the Integral Test.}} \fi

\item $\sum_{k=1}^{\infty}{\left(\arctan\left(\frac{1}{k}\right)-\arctan(k)\right)}$

\ifans{\fbox{The series diverges by the Divergence Test. [see \underline{Sequences} problem \#33.]}} \fi

\item $\sum_{k=1}^{\infty}{\frac{1}{\sqrt[4]{k+15}}}$

\ifans{\fbox{$\sum_{k=1}^{\infty}{\frac{1}{\sqrt[4]{k+15}}}=\sum_{k=16}^{\infty}{\frac{1}{\sqrt[4]{k}}}$, which is a $p$-series with $p=\frac{1}{4}<1$, so the series diverges.   }} \fi

\item $\sum_{k=1}^{\infty}{\pi^k e^{-k}}$

\ifans{\fbox{\parbox{1\linewidth}{The series diverges by the Divergence Test. Also, observe that this is a geometric series with ratio $r=\textstyle \frac{\pi}{e}>1$, which confirms that the series diverges.}}} \fi

\item $\sum_{k=2}^{\infty}{\frac{1}{4k^2}}$

\ifans{\fbox{The series is a constant multiple of a $p$-series with $p=2>1$, so the series converges.}} \fi

\item $\sum_{k=2}^{\infty}{\frac{k^2}{4k^2+9}}$

\ifans{\fbox{The series diverges by the Divergence Test.}} \fi

\item $\sum_{k=2}^{\infty}{\frac{k}{4k^2+9}}$

\ifans{\fbox{The series diverges by the Integral Test.}} \fi

\item $\sum_{k=2}^{\infty}{\frac{1}{4k^2+9}}$

\ifans{\fbox{The series converges by the Integral Test.; Detailed Solution: \textcolor{blue}{\href{http://www.math.drexel.edu/classes/Calculus/resources/Math123HW/Solutions/123_08_Convergence_Tests_18.pdf}{Here}}}} \fi

\item $\sum_{k=2}^{\infty}{\frac{1}{4k^2-9}}$

\ifans{\fbox{The series converges by the Integral Test.; Detailed Solution: \textcolor{blue}{\href{http://www.math.drexel.edu/classes/Calculus/resources/Math123HW/Solutions/123_08_Convergence_Tests_19.pdf}{Here}}}} \fi

\item $\sum_{k=10}^{\infty}{15k^{-0.999}}$

\ifans{\fbox{The series is a constant multiple of a $p$-series with $p=0.999<1$, so the series diverges.}} \fi

\end{enumerate}

\noindent{\bf For problems 21 \& 22, use algebraic properties of series to find the sum of the series.}

\begin{enumerate}
\setcounter{enumi}{20}

\item $\sum_{k=1}^{\infty}{\left[\frac{1}{6^k}-\left(\frac{1}{k}-\frac{1}{k+1}\right)\right]   }$

\ifans{\fbox{$-\frac{4}{5}$ }} \fi

\item $\frac{1}{2}+2-\frac{1}{4}+\frac{4}{7}+\frac{1}{8}+\frac{8}{49}-\frac{1}{16}+\frac{16}{343}+\ldots$  \newline  \newline [Hint: See \underline{Infinite Series} problems \#11 \& \#12.]

\ifans{\fbox{$\frac{47}{15}$ }} \fi

\end{enumerate}

\end{document}