\documentclass[12pt]{article}


\usepackage{amssymb}
\usepackage{amsmath}
\usepackage{fullpage}
\usepackage{epsfig}
\usepackage{epstopdf}
\everymath{\displaystyle}
\usepackage{enumerate}
\usepackage[hidelinks]{hyperref}
\usepackage{enumitem}
\usepackage{xcolor}




\begin{document}

\begin{center}
\underline{\LARGE{Sequences}}
\end{center}

\noindent SUGGESTED REFERENCE MATERIAL:

\medskip

\noindent As you work through the problems listed below, you should reference your lecture notes and the relevant chapters in a textbook/online resource.

\medskip

\noindent EXPECTED SKILLS:

\medskip

\begin{itemize}[topsep=0pt]

\item Find the general term of a sequence.

\item Determine whether a sequence converges, and if so, what it converges to.  This may require techniques such as L'Hopital's Rule and The Squeeze Theorem.

\end{itemize}

\bigskip

\noindent PRACTICE PROBLEMS:

\medskip

\noindent {\bf For problems 1 -- 8, rewrite the sequence by placing the general term inside braces.}

\begin{enumerate}

\item $\frac{1}{4}, \frac{1}{16}, \frac{1}{64}, \frac{1}{256}, ...$

\includegraphics[scale=0.5]{start.pdf}
{{{1\linewidth}{ $\left\{ \frac{1}{4^n} \right\}_{n = 1}^{+\infty}$.  Note that any integer can be used as the lower index for a sequence.  So you could describe this sequence with 
$\left\{ \frac{1}{4^{n+1}} \right\}_{n = 0}^{+\infty}$ or even $\left\{ \frac{1}{4^{n+10}} \right\}_{n = -9}^{+\infty}$, although admittedly it is weird to use $-9$ as a starting index unless 
there is some compelling reason to do so.  It is recommended to use either 0 or 1 as a starting index in most cases.  }}}
\includegraphics[scale=0.5]{end.pdf}


\item $\frac{1}{4}, -\frac{1}{16}, \frac{1}{64}, -\frac{1}{256},...$

\includegraphics[scale=0.5]{start.pdf}
{{ $\left\{ (-1)^{n+1} \text{ } \frac{1}{4^n} \right\}_{n = 1}^{+\infty}$  }}
\includegraphics[scale=0.5]{end.pdf}


\item $0, 1, 2^3, 3^4, 4^5, ...$

\includegraphics[scale=0.5]{start.pdf}
{{ $\left\{ n^{n+1} \right\}_{n = 0}^{+\infty}$  }}
\includegraphics[scale=0.5]{end.pdf}


\item $3, 2, 1, 0, -1, -2, -3, -4, -5, ...$

\includegraphics[scale=0.5]{start.pdf}
{{ $\left\{ 4-n \right\}_{n = 1}^{+\infty}$  }}
\includegraphics[scale=0.5]{end.pdf}


\item $1, \frac{1}{e}, e^2, \frac{1}{e^3}, e^4, \frac{1}{e^5}, ...$

\includegraphics[scale=0.5]{start.pdf}
{{ $\left\{  e^{(-1)^nn}  \right\}_{n = 0}^{+\infty}$  }}
\includegraphics[scale=0.5]{end.pdf}


\item $\frac{1}{1 \cdot 2}, \frac{1}{1 \cdot 2 \cdot 3 \cdot 4}, \frac{1}{1 \cdot 2 \cdot 3 \cdot 4 \cdot 5 \cdot 6}, \frac{1}{1 \cdot 2 \cdot 3 \cdot 4 \cdot 5 \cdot 6 \cdot 7 \cdot 8}, ...$

\includegraphics[scale=0.5]{start.pdf}
{{ $\left\{  \frac{1}{(2n)!} \right\}_{n = 1}^{+\infty}$  }}
\includegraphics[scale=0.5]{end.pdf}


\item $\frac{3}{1 \cdot 2}, \frac{5}{1 \cdot 2 \cdot 3 \cdot 4}, \frac{7}{1 \cdot 2 \cdot 3 \cdot 4 \cdot 5 \cdot 6}, \frac{9}{1 \cdot 2 \cdot 3 \cdot 4 \cdot 5 \cdot 6 \cdot 7 \cdot 8}, ...$

\includegraphics[scale=0.5]{start.pdf}
{{ $\left\{  \frac{2n+1}{(2n)!} \right\}_{n = 1}^{+\infty}$  }}
\includegraphics[scale=0.5]{end.pdf}


\item $0, 1, 0, -1, 0, 1, 0, -1, 0, 1 ...$ [Hint: Think about a trigonometric function.]

\includegraphics[scale=0.5]{start.pdf}
{{ $\left\{ \sin{  \Bigl( \frac{\pi}{2}}n \Bigr)  \right\}_{n = 0}^{+\infty}$ or  $\left\{ -\cos{  \Bigl( \frac{\pi}{2}}n \Bigr)  \right\}_{n = 1}^{+\infty}$ or
$\left\{ \cos{  \Bigl( \frac{\pi}{2}}n \Bigr)  \right\}_{n = -1}^{+\infty}$.  There are other possibilities as well.  }}
\includegraphics[scale=0.5]{end.pdf}


\item For each of the sequences in problems 1 -- 8, determine if the sequence converges, and if so, what it converges to.  If it diverges, determine if the general term approaches $+\infty$, $-\infty$, or neither.

\includegraphics[scale=0.5]{start.pdf}
{{{1\linewidth}
{The sequence in problem:
\begin{itemize} 
\item[] \#1 converges to 0.
\item[] \#2 converges to 0.
\item[] \#3 diverges to $+\infty$.
\item[] \#4 diverges to $-\infty$.
\item[] \#5 diverges.  The even-numbered terms converge to $0$, but the odd-numbered terms diverge to $+\infty$
\item[] \#6 converges to 0.
\item[] \#7 converges to 0.
\item[] \#8 diverges.  The odd-numbered terms converge to $0$ (in fact, they are all equal to $0$), but the even-numbered terms diverge since they oscillate between $1$ and $-1$. 
\end{itemize}
Detailed Solution: \textcolor{blue}{\href{http://www.math.drexel.edu/classes/Calculus/resources/Math123HW/Solutions/123_05_Sequences_09.pdf}{Here}}}}}
\includegraphics[scale=0.5]{end.pdf}


\end{enumerate}

\noindent {\bf For problems 10 -- 35, determine if the sequence converges, and if so, what it converges to.  If it diverges, determine if the general term approaches $+\infty$, $-\infty$, or neither.}
 
\begin{enumerate}
\setcounter{enumi}{9}
  
\item $\left\{ 5 \right\}_{n = 1}^{+\infty}$

\includegraphics[scale=0.5]{start.pdf}
{{Converges to $5$.}}
\includegraphics[scale=0.5]{end.pdf}


\item $\left\{ 5n \right\}_{n = 0}^{+\infty}$

\includegraphics[scale=0.5]{start.pdf}
{{Diverges to $+\infty$.}}
\includegraphics[scale=0.5]{end.pdf}


\item $\left\{ 5-5n^3 \right\}_{n = 1}^{+\infty}$

\includegraphics[scale=0.5]{start.pdf}
{{Diverges to $-\infty$.}}
\includegraphics[scale=0.5]{end.pdf}


\item $\left\{ \frac{4n-3n^5}{2n^5+4n^3+n^2+5} \right\}_{n = 1}^{+\infty}$

\includegraphics[scale=0.5]{start.pdf}
{{Converges to $-\frac{3}{2}$.  See \underline{Limits at Infinity Review} problem \#2.}}
\includegraphics[scale=0.5]{end.pdf}


\item $\left\{ (-1)^n \text{ } \frac{n^3+n^2+n+1}{n^3+1} \right\}_{n = 1}^{+\infty}$

\includegraphics[scale=0.5]{start.pdf}
{{{1\linewidth}{The odd-numbered terms converge to $-1$, but the even-numbered terms converge to $1$, so the sequence diverges.}}}
\includegraphics[scale=0.5]{end.pdf}


\item $\left\{ \frac{n^4-3n^3-2n}{4n^2+19} \right\}_{n = 0}^{+\infty}$

\includegraphics[scale=0.5]{start.pdf}
{{Diverges to $+\infty$.}}
\includegraphics[scale=0.5]{end.pdf}


\item $\left\{ \frac{1-10n^2}{n^2-4n^3} \right\}_{n = 1}^{+\infty}$

\includegraphics[scale=0.5]{start.pdf}
{{Converges to $0$.}}
\includegraphics[scale=0.5]{end.pdf}


\item $\left\{ (-1)^{n+1} \text{ } \frac{1-10n^2}{n^2-4n^3} \right\}_{n = 1}^{+\infty}$

\includegraphics[scale=0.5]{start.pdf}
{{Converges to $0$.}}
\includegraphics[scale=0.5]{end.pdf}


\item $\left\{ \frac{\sqrt{4+3n^2}}{2+7n} \right\}_{n = 1}^{+\infty}$

\includegraphics[scale=0.5]{start.pdf}
{{Converges to $\frac{\sqrt{3}}{7}$.  See \underline{Limits at Infinity Review} problem \#3.}}
\includegraphics[scale=0.5]{end.pdf}


\item $\left\{ e^{1/n} \right\}_{n = 1}^{+\infty}$

\includegraphics[scale=0.5]{start.pdf}
{{Converges to $1$.}}
\includegraphics[scale=0.5]{end.pdf}


\item $\left\{ \frac{e^{-n}}{n^{-2}} \right\}_{n = 1}^{+\infty}$

\includegraphics[scale=0.5]{start.pdf}
{{Converges to $0$.}}
\includegraphics[scale=0.5]{end.pdf}


\item $\left\{ \frac{e^n-e^{-n}}{e^n+e^{-n}} \right\}_{n = 1}^{+\infty}$

\includegraphics[scale=0.5]{start.pdf}
{{Converges to $1$.  See \underline{Limits at Infinity Review} problem \#5.; Detailed Solution: \textcolor{blue}{\href{http://www.math.drexel.edu/classes/Calculus/resources/Math123HW/Solutions/123_05_Sequences_21.pdf}{Here}}}}
\includegraphics[scale=0.5]{end.pdf}



\item $\left\{ \frac{e^{\sqrt{n}}}{n} \right\}_{n = 1}^{+\infty}$

\includegraphics[scale=0.5]{start.pdf}
{{Diverges to $+\infty$.}}
\includegraphics[scale=0.5]{end.pdf}


\item $\left\{e^n \sin(e^{-n}) \right\}_{n = 1}^{+\infty}$ 

\includegraphics[scale=0.5]{start.pdf}
{{Converges to $1$.}}
\includegraphics[scale=0.5]{end.pdf}


\item $\left\{ e^n \pi^{-n} \right\}_{n = 1}^{+\infty}$

\includegraphics[scale=0.5]{start.pdf}
{{Converges to $0$.}}
\includegraphics[scale=0.5]{end.pdf}


\item $\left\{ \ln{\left(\frac{1}{n}\right)} \right\}_{n = 1}^{+\infty}$

\includegraphics[scale=0.5]{start.pdf}
{{Diverges to $-\infty$.}}
\includegraphics[scale=0.5]{end.pdf}


\item $\left\{ \frac{\ln{(6n)}}{\ln{(2n)}} \right\}_{n = 1}^{+\infty}$

\includegraphics[scale=0.5]{start.pdf}
{{Converges to $1$.; Detailed Solution: \textcolor{blue}{\href{http://www.math.drexel.edu/classes/Calculus/resources/Math123HW/Solutions/123_05_Sequences_26.pdf}{Here}}}}
\includegraphics[scale=0.5]{end.pdf}


\item $\left\{ \ln{(n+2)} - \ln{(3n+5)} \right\}_{n = 0}^{+\infty}$

\includegraphics[scale=0.5]{start.pdf}
{{Converges to $\ln{\left(\frac{1}{3}\right)}$.}}
\includegraphics[scale=0.5]{end.pdf}


\item $\left\{ \sqrt{n^2+8n-5}-n \right\}_{n = 1}^{+\infty}$

\includegraphics[scale=0.5]{start.pdf}
{{Converges to $4$.  See \underline{Limits at Infinity Review} problem \#4.; Detailed Solution: \textcolor{blue}{\href{http://www.math.drexel.edu/classes/Calculus/resources/Math123HW/Solutions/123_05_Sequences_28.pdf}{Here}}}}
\includegraphics[scale=0.5]{end.pdf}



\item $\left\{ \sqrt{n^2-n} + n \right\}_{n = 0}^{+\infty}$

\includegraphics[scale=0.5]{start.pdf}
{{Diverges to $+\infty$.}}
\includegraphics[scale=0.5]{end.pdf}


\item $\left\{ \sqrt{n^2-n} - n \right\}_{n = 1}^{+\infty}$

\includegraphics[scale=0.5]{start.pdf}
{{Converges to $-\frac{1}{2}$.}}
\includegraphics[scale=0.5]{end.pdf}


\item $\left\{ \frac{\cos{n}}{n} \right\}_{n = 1}^{+\infty}$

\includegraphics[scale=0.5]{start.pdf}
{{Converges to $0$.  See \underline{Limits at Infinity Review} problem \#7.}}
\includegraphics[scale=0.5]{end.pdf}


\item $\left\{\arccos\left(\frac{n^2}{3n-n^2}\right) \right\}_{n = 1}^{+\infty}$

\includegraphics[scale=0.5]{start.pdf}
{{Converges to $\pi$.}}
\includegraphics[scale=0.5]{end.pdf}


\item $\left\{ \arctan\left({\frac{1}{n}}\right) - \arctan{(n)}  \right\}_{n = 1}^{+\infty}$

\includegraphics[scale=0.5]{start.pdf}
{{Converges to $-\frac{\pi}{2}$.  See \underline{Limits at Infinity Review} problem \#8.}}
\includegraphics[scale=0.5]{end.pdf}


\item $\left\{ \left(1+\frac{1}{n}\right)^n \right\}_{n = 1}^{+\infty}$

\includegraphics[scale=0.5]{start.pdf}
{{Converges to $e$.  See \underline{Limits at Infinity Review} problem \#9.}}
\includegraphics[scale=0.5]{end.pdf}


\item $\left\{ \left(1+3^n\right)^{1/n} \right\}_{n = 1}^{+\infty}$

\includegraphics[scale=0.5]{start.pdf}
{{Converges to $3$.  See \underline{Limits at Infinity Review} problem \#10.; Detailed Solution: \textcolor{blue}{\href{http://www.math.drexel.edu/classes/Calculus/resources/Math123HW/Solutions/123_05_Sequences_35.pdf}{Here}}}}
\includegraphics[scale=0.5]{end.pdf}



\item $\left\{ \left(\frac{4}{n}\right)^{2/n} \right\}_{n = 1}^{+\infty}$

\includegraphics[scale=0.5]{start.pdf}
{{Converges to $1$.}}
\includegraphics[scale=0.5]{end.pdf}


\item Consider the sequence $\sqrt{30}, \sqrt{30+\sqrt{30}}, \sqrt{30+\sqrt{30+\sqrt{30}}}, ...$

\begin{enumerate}

\item Define the sequence recursively.

\includegraphics[scale=0.5]{start.pdf}
{{$a_1=\sqrt{30}, a_{n+1}=\sqrt{30+a_n}$ for integers $n\geq1$.}}
\includegraphics[scale=0.5]{end.pdf}


\item Assuming the sequence converges to some limit $L$, find $L$.

\includegraphics[scale=0.5]{start.pdf}
{{$L=6$.}}
\includegraphics[scale=0.5]{end.pdf}


\end{enumerate}

\item Consider the sequence $\left\{ a_n \right\}_{n = 1}^{+\infty}$ that has the following recursive definition: \newline $a_{n+1}=10-a_n$ for integers $n\geq1$.

\begin{enumerate}

\item Assuming the sequence converges to some limit $L$, find $L$.

\includegraphics[scale=0.5]{start.pdf}
{{$L=5$.}}
\includegraphics[scale=0.5]{end.pdf}


\item How must $a_1$ be defined to ensure that the sequence converges?  Justify your answer.

\includegraphics[scale=0.5]{start.pdf}
{{{1\linewidth}{If $a_1 = 5$, the sequence is $5, 5, 5, 5, ...$, which clearly converges to 5.  If $a_1 \neq 5$, say $a_1 = K (K \neq 5)$, then the sequence oscillates bewteen $K$ and $10-K$, e.g. 
if $a_1 = 3$ we have $3, 7, 3, 7, 3, 7, ...$  Such a sequence diverges.}}}
\includegraphics[scale=0.5]{end.pdf}


\end{enumerate}

\item The \underline{Fibonacci sequence} $1, 1, 2, 3, 5, 8, 13, 21, ...$ begins with two $1$'s and thereafter each term in the sequence is the the sum of previous two terms.

\begin{enumerate}

\item Define the Fibonacci sequence recursively.

\includegraphics[scale=0.5]{start.pdf}
{{$a_1=1, a_2=1, a_{n+2}=a_n + a_{n+1}$ for integers $n\geq1$.}}
\includegraphics[scale=0.5]{end.pdf}


\item Clearly the Fibonacci sequence diverges to $+\infty$, but consider the ratio of successive terms $\frac{a_{n+1}}{a_n}$ for $n\geq1$, i.e $$\frac{1}{1}, \frac{2}{1}, \frac{3}{2}, \frac{5}{3}, 
\frac{8}{5}, ...$$  Assuming this ``ratio sequence'' converges to some limit $L$, find $L$.

\includegraphics[scale=0.5]{start.pdf}
{{{1\linewidth}{ $L=\frac{1+\sqrt{5}}{2}$.  This number is known as the \underline{Golden Ratio}. \\
Fibonacci Fun Fact: The Fibonacci Sequence can be described without recursion as 
$$\left\{ \frac{ \left(\frac{1+\sqrt{5}}{2}\right)^n -  \left(\frac{1-\sqrt{5}}{2}\right)^n} {\sqrt{5}}\right\}_{n = 1}^{+\infty},$$  and with some clever factoring it can be shown from this non-recursive definition
that the limit of the ``ratio sequence'' is $L=\frac{1+\sqrt{5}}{2}$. \\ Detailed Solution: \textcolor{blue}{\href{http://www.math.drexel.edu/classes/Calculus/resources/Math123HW/Solutions/123_05_Sequences_39.pdf}{Here}}}}}
\includegraphics[scale=0.5]{end.pdf}




\end{enumerate}

\end{enumerate}

\end{document}