\documentclass[12pt]{article}


\usepackage{amssymb}
\usepackage{amsmath}
\usepackage{fullpage}
\usepackage{epsfig}
\usepackage{epstopdf}
\everymath{\displaystyle}
\usepackage{enumerate}
\usepackage[hidelinks]{hyperref}
\usepackage{enumitem}
\usepackage{xcolor}

\newif\ifans

\ansfalse

\begin{document}

\begin{center}
\underline{\LARGE{Monotone Sequences}}
\end{center}

\noindent SUGGESTED REFERENCE MATERIAL:

\medskip

\noindent As you work through the problems listed below, you should reference your lecture notes and the relevant chapters in a textbook/online resource.

\medskip

\noindent EXPECTED SKILLS:

\medskip

\begin{itemize}[topsep=0pt]

\item Understand what it means for a sequence to be increasing, decreasing, strictly increasing, strictly decreasing, eventually increasing, or eventually decreasing. 

\item Use an approriate test for monotonicity to determine if a sequence is increasing or decreasing.

\item Show that a sequence must converge to a limit by showing that it is montone and appropriately bounded.

\end{itemize}

\bigskip

\noindent PRACTICE PROBLEMS:

\begin{enumerate}

\item Give an example of a convergent sequence that is not a monotone sequence. 

\ifans{\fbox{One possibility is $\left\{ (-1)^n \text{ } \frac{1}{n} \right\}_{n = 1}^{+\infty}=-1, \frac{1}{2}, -\frac{1}{3}, \frac{1}{4}, ... $ , which converges to $0$ but is not monotonic.}} \fi

\item Give an example of a sequence that is bounded from above and bounded from below but is not convergent.

\ifans{\fbox{\parbox{1\linewidth}{One possibility is $\left\{ (-1)^n \right\}_{n = 1}^{+\infty}=-1, 1, -1, 1, -1, 1 ... $, which is bounded from above by $1$ (or any number greater than $1$) and is bounded below by $-1$
(or any number less than $-1$).  However, the sequence diverges since its terms oscillate between $1$ and $-1$.}}} \fi


\end{enumerate}

\noindent {\bf For problems 3 and 4, determine if the sequence is increasing or decreasing by calculating $a_{n+1}-a_n$.}

\begin{enumerate}
\setcounter{enumi}{2}

\item $\left\{ \frac{1}{4^n} \right\}_{n = 1}^{+\infty}$

\ifans{\fbox{The sequence is (strictly) decreasing.}} \fi

\item $\left\{ \frac{2n-3}{3n-2} \right\}_{n = 1}^{+\infty}$

\ifans{\fbox{The sequence is (strictly) increasing.}} \fi

\end{enumerate}

\noindent {\bf For problems 5 amd 6, determine if the sequence is increasing or decreasing by calculating $\frac{a_{n+1}}{a_n}$.}

\begin{enumerate}
\setcounter{enumi}{4}

\item $\left\{ \frac{1}{4^n} \right\}_{n = 1}^{+\infty}$

\ifans{\fbox{The sequence is (strictly) decreasing.}} \fi

\item $\left\{ \frac{e^n-e^{-n}}{e^n+e^{-n}} \right\}_{n = 1}^{+\infty}$

\ifans{\fbox{The sequence is (strictly) increasing.; Detailed Solution: \textcolor{blue}{\href{http://www.math.drexel.edu/classes/Calculus/resources/Math123HW/Solutions/123_06_Monotone_Sequences_06.pdf}{Here}}}} \fi


\end{enumerate}

\noindent {\bf For problems 7 and 8, determine if the sequence is increasing or decreasing by calculating the derivative $a_n^\prime$.}

\begin{enumerate}
\setcounter{enumi}{6}

\item $\left\{ \frac{1}{4^n} \right\}_{n = 1}^{+\infty}$

\ifans{\fbox{The sequence is (strictly) decreasing.}} \fi

\item $\left\{ \frac{\ln(2n)}{\ln(6n)} \right\}_{n = 1}^{+\infty}$

\ifans{\fbox{The sequence is (strictly) increasing.}} \fi

\end{enumerate}

\noindent {\bf For problems 9 -- 17, use an appropriate test for monotonicity to determine if the sequence increases, decreases, eventually increases, or eventually decreases.}
 
\begin{enumerate}
\setcounter{enumi}{8}
  
\item $\left\{ \frac{3n}{2n+1} \right\}_{n = 1}^{+\infty}$

\ifans{\fbox{The sequence is (strictly) increasing.}} \fi

\item $\left\{ n - \frac{1}{n} \right\}_{n = 1}^{+\infty}$

\ifans{\fbox{The sequence is (strictly) increasing.}} \fi

\item $\left\{ \frac{n^2}{n!} \right\}_{n = 1}^{+\infty}$

\ifans{\fbox{The sequence is eventually (strictly) decreasing.}} \fi

\item $\left\{ \frac{2n+1}{(2n)!} \right\}_{n = 1}^{+\infty}$

\ifans{\fbox{The sequence is (strictly) decreasing.; Detailed Solution: \textcolor{blue}{\href{http://www.math.drexel.edu/classes/Calculus/resources/Math123HW/Solutions/123_06_Monotone_Sequences_12.pdf}{Here}}}} \fi


\item $\left\{ \frac{e^{\sqrt{n}}}{n} \right\}_{n=1}^{+\infty}$

\ifans{\fbox{The sequence is eventually (strictly) increasing.}} \fi


\item $\left\{ e^n \pi^{-n} \right\}_{n = 1}^{+\infty}$

\ifans{\fbox{The sequence is (strictly) decreasing.}} \fi

\item $\left\{ \frac{3^{(n^2)}}{(1000)^n} \right\}_{n = 1}^{+\infty}$

\ifans{\fbox{The sequence is eventually (strictly) increasing.}} \fi

\item $\left\{ \frac{n!}{n^n} \right\}_{n = 1}^{+\infty}$

\ifans{\fbox{The sequence is (strictly) decreasing.}} \fi

\item $\left\{ n^3 e^{-n} \right\}_{n = 1}^{+\infty}$

\ifans{\fbox{The sequence is eventually (strictly) decreasing.; Detailed Solution: \textcolor{blue}{\href{http://www.math.drexel.edu/classes/Calculus/resources/Math123HW/Solutions/123_06_Monotone_Sequences_17.pdf}{Here}}}} \fi


\item In the previous set of assigned problems it was shown that {\bf if} the sequence $$\sqrt{30}, \sqrt{30+\sqrt{30}}, \sqrt{30+\sqrt{30+\sqrt{30}}}, ... $$ converged to a limit, that limit was  $6$.  Now we will
show that the sequence is bounded above and increasing; thus, it must converge.

\begin{enumerate}

\item Define the sequence recursively.

\ifans{\fbox{$a_1=\sqrt{30}, a_{n+1}=\sqrt{30+a_n}$ for integers $n\geq1$.}} \fi

\item Show that the sequence has an upper bound of $6$.

\ifans{\fbox{\parbox{1\linewidth}{$a_1=\sqrt{30}<\sqrt{36}=6$, so $a_1<6$.
\\ $a_2=\sqrt{30+a_1}<\sqrt{30+6}=6$, so $a_2<6$.
\\ $a_3=\sqrt{30+a_2}<\sqrt{30+6}=6$, so $a_3<6$.
\\ This continues indefinitely, so $a_n<6$ for all integers $n\geq1$, i.e. the sequence is bounded from above by 6.  (It is also bounded from below by 0). }}} \fi

\item Show that the sequence is increasing by computing $a_{n+1}^2-a_n^2$.

\ifans{\fbox{\parbox{1\linewidth}{$a_{n+1}^2-a_n^2 = 30 + a_n - a_n^2 = (5+a_n)(6-a_n)$. 
\\ Now from part (b) $0<a_n<6$, so $5+a_n>0$ and $6-a_n>0$, so $a_{n+1}^2-a_n^2 > 0$.
\\ Also, $a_{n+1}^2-a_n^2=(a_{n+1}-a_n)(a_{n+1}+a_n)$, so $(a_{n+1}-a_n)(a_{n+1}+a_n) > 0$.
\\ Since every term in the sequence is positive, we now have $(a_{n+1}-a_n)>0$, or $a_{n+1}>a_n$, i.e. the sequence is (strictly) increasing.    }}} \fi

\end{enumerate}

\end{enumerate}

\end{document}