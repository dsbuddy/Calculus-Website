\documentclass[12pt]{article}


\usepackage{amssymb}
\usepackage{amsmath}
\usepackage{fullpage}
\usepackage{epsfig}
\usepackage{epstopdf}
\everymath{\displaystyle}

\newif\ifans

\ansfalse

\begin{document}

\begin{center}
\underline{\LARGE{Chapter 1.2 Practice Problems}}
\end{center}

\noindent EXPECTED SKILLS:

\begin{itemize}

\item Know the basic properties of limits; i.e., be familiar with how limits "interact" with sums, differences, products, and other operations.  See Theorem 1.2.2.

\item Given the formula of a function $y=f(x)$, be able to determine the limit of $f(x)$ as $x$ approaches some finite value (as both a one-sided and two sided limit).

\item Be able to determine when such a limit does not exist, and if appropiate, indicate if the behavior of the function is increasing or decreasing without bound.

\item Be familiar with the indeterminate forms of $\displaystyle \frac{0}{0}$ and $\displaystyle \frac{\pm \infty}{\pm \infty}$.  And, know how to use algebraic techniques such as factoring and rationalizing to help compute these types of limits.

\end{itemize}

\noindent PRACTICE PROBLEMS:

\noindent {\bf In each problem, compute the limit. If the limit doesn't exist write $+\infty$, $-\infty$, or DNE (whichever is most appropriate).}

\begin{enumerate}

\item Given that $\displaystyle \lim_{x\rightarrow 1}{f(x)} = 4$ and  $\displaystyle \lim_{x\rightarrow 1}{g(x)}=2 $, determine each of the following limits:

\begin{enumerate}

\item $\displaystyle \lim_{x\rightarrow 1}{\left(f(x)+g(x)\right)}$

\ifans{\fbox{6}} \fi

\item $\displaystyle \lim_{x\rightarrow 1}{\left(5f(x)-g(x)\right)}$

\ifans{\fbox{$18$}} \fi

\item $\displaystyle \lim_{x\rightarrow 1}{\left(\frac{f(x)}{g(x)}\right)}$

\ifans{\fbox{2}} \fi

\end{enumerate}

\item  $\displaystyle \lim_{x\rightarrow 1}{\left(x^2+1\right)}$

\ifans{\fbox{2}} \fi

\item  $\displaystyle \lim_{x\rightarrow 4}{1}$

\ifans{\fbox{1}} \fi

\item $\displaystyle \lim_{x\rightarrow -1}{(x+1)(x^3)}$

\ifans{\fbox{0}} \fi

\item  $\displaystyle \lim_{x\rightarrow 5^-}{\left(\frac{x^2-6x}{x^3-1}\right)}$

\ifans{\fbox{$\displaystyle -\frac{5}{124}$}} \fi

\item $\displaystyle \lim_{x\rightarrow -1}{\left(\frac{x^2-1}{x+1}\right)}$ 

\ifans{\fbox{$-2$}} \fi
 
\item $\displaystyle \lim_{x\rightarrow 2^-}{\left(\frac{x^2-4x+4}{x-2}\right)}$ 

\ifans{\fbox{0}} \fi
 
 \item $\displaystyle \lim_{x\rightarrow 3^+}{\left(\frac{x^2+2x-15}{x-3}\right)}$

\ifans{\fbox{8}} \fi

\item $\displaystyle \lim_{x \rightarrow 1}{\left(\frac{x^3-3x^2-x+3}{x^2-1}\right)}$

\ifans{\fbox{$-2$}} \fi

\item $\displaystyle \lim_{x\rightarrow 16}{\left(\frac{\sqrt{x}-4}{x-16}\right)}$

\ifans{\fbox{$\displaystyle \frac{1}{8}$}} \fi

\item $\lim_{x \rightarrow 0}\left(\frac{|x|}{x}\right)$

\ifans{\fbox{DNE}} \fi

\item $\displaystyle \lim_{x\rightarrow 4^-}{\left(\frac {x}{x-4}\right)}$

\ifans{\fbox{$-\infty$}} \fi

\item  $\displaystyle \lim_{x\rightarrow 4^+}{\left(\frac{x}{x-4}\right)}$

\ifans{\fbox{$+\infty$}} \fi

\item $\displaystyle \lim_{x\rightarrow 4}{\left(\frac{x}{x-4}\right)}$

\ifans{\fbox{DNE}} \fi

\item $\displaystyle \lim_{x\rightarrow -2}{\left(\frac{1}{x-2}\right)}$

\ifans{\fbox{$\displaystyle -\frac{1}{4}$}} \fi

\item $\displaystyle \lim_{x\rightarrow -2^-}{\left(\frac{x}{x^2+2x}\right)}$

\ifans{\fbox{$-\infty$}} \fi

\item $\displaystyle \lim_{x\rightarrow -2^+}{\left(\frac{x}{x^2+2x}\right)}$

\ifans{\fbox{$+\infty$}} \fi
 
\item $\displaystyle \lim_{x\rightarrow 3}{\left(\frac{x^3}{|x-3|}\right)}$

\ifans{\fbox{$+\infty$}} \fi

\item $\displaystyle \lim_{x \rightarrow 1^-}{\left(\frac{x-1}{x^2-2x+1}\right)}$

\ifans{\fbox{$-\infty$}} \fi

\item $\displaystyle \lim_{x \rightarrow 1^+}{\left(\frac{x-1}{x^2+2x-3}\right)}$

\ifans{\fbox{$\displaystyle \frac{1}{4}$}} \fi

\item Let $\displaystyle 
      f(n) = \begin{cases}
      n^2+1, & \text{if }n \leq -1 \\
      3n+1, & \text{if }n > -1 \end{cases}$. Compute $\displaystyle \lim_{n\rightarrow -1}{f(n)}$

\ifans{\fbox{DNE}} \fi

\item Let $\displaystyle 
        f(x) = \begin{cases}
        3x^3+2x-3, & \text{if }x < 1   \\
        100, & \text{if }x = 1 \\
        \frac{x^2-1}{x-1}, & \text{if }x > 1 \end{cases}$. Determine $\displaystyle \lim_{x\rightarrow 1}{f(x)}$ 

\ifans{\fbox{2}} \fi

\item Let $\displaystyle f(p) = \begin{cases} 
        3p-1, & \text{if } p<3 \\
        p^3-4p-7, & \text{if } p>3 \end{cases}$. Find $\displaystyle \lim_{p\rightarrow 3}{f(p)}$

\ifans{\fbox{8}} \fi

\item Let $f(x) = \begin{cases}
        x^2+2ax+a^2, & \text{if } x < 4 \\
        432, & \text{if } x = 4 \\
        x^2 -7, & \text{if } x > 4 \end{cases}$. Find the value(s) of $a$ such that $\displaystyle \lim_{x\rightarrow 4}{f(x)}$ exists. 

\ifans{\fbox{$-1$ or $-7$}} \fi

\item Let $x_0$ be a fixed real number.  Compute $\displaystyle \lim_{h \rightarrow 0}{\frac{(x_0+h)^2-x_0^2}{h}}$

\ifans{\fbox{$2x_0$}} \fi

\item Let $x_0$ be a fixed real number.  Compute $\displaystyle \lim_{x \rightarrow x_0}{\frac{x^2-x_0^2}{x-x_0}}$

\ifans{\fbox{$2x_0$}} \fi

\item Let $x_0$ be a fixed, positive real number.  Compute $\displaystyle \lim_{h \rightarrow 0}{\frac{\sqrt{x_0+h}-\sqrt{x_0}}{h}}$

\ifans{\fbox{$\displaystyle \frac{1}{2\sqrt{x_0}}$}} \fi

\end{enumerate}

\end{document}