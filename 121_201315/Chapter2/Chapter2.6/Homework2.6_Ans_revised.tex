\documentclass[12pt]{article}


\usepackage{amssymb}
\usepackage{amsmath}
\usepackage{fullpage}
\usepackage{epsfig}
\usepackage{epstopdf}
\everymath{\displaystyle}



\begin{document}

\begin{center}
\underline{\LARGE{Chapter 2.6 Practice Problems}}
\end{center}

\noindent EXPECTED SKILLS:

\begin{itemize}

\item Know how to use the chain rule to calculate derivatives of compositions of functions.

\end{itemize}

\noindent PRACTICE PROBLEMS:

\medskip

\noindent {\bf For problems 1-16, calculate the derivative of the given function.}

\begin{enumerate}

\item $f(x) = (x^3+4)^{-3}$ 

\includegraphics[scale=0.5]{start.pdf}
{{$-9x^2(x^3+4)^{-4}$}}
\includegraphics[scale=0.5]{end.pdf}


\item $f(x) = (x^2+2x)^6$ 

\includegraphics[scale=0.5]{start.pdf}
{{$6(2x+2)(x^2+2x)^5$}}
\includegraphics[scale=0.5]{end.pdf}


\item $f(x) = \sqrt{x^3-2}$ 

\includegraphics[scale=0.5]{start.pdf}
{{$\frac{3x^2}{2\sqrt{x^3-2}}$}}
\includegraphics[scale=0.5]{end.pdf}


\item $f(x) = \tan{\left(\frac{1}{x^2}\right)}$ 

\includegraphics[scale=0.5]{start.pdf}
{{$-2x^{-3}\sec^2{\left(\frac{1}{x^2}\right)}$}}
\includegraphics[scale=0.5]{end.pdf}


\item $f(x) = \sec{2x}$ 

\includegraphics[scale=0.5]{start.pdf}
{{$2\sec{(2x)}\tan{(2x)}$}}
\includegraphics[scale=0.5]{end.pdf}


\item $f(x) = \cos^3{3x}$ 

\includegraphics[scale=0.5]{start.pdf}
{{$-9\sin{(3x)}\cos^2{(3x)}$}}
\includegraphics[scale=0.5]{end.pdf}


\item $f(x) = \left(x^5-\frac{1}{x^2}\right)^4$ 

\includegraphics[scale=0.5]{start.pdf}
{{$4\left(x^5-\frac{1}{x^2}\right)^3\left(5x^4+\frac{2}{x^3}\right)$}}
\includegraphics[scale=0.5]{end.pdf}


\item $f(x) = \frac{x^2-3}{(3x-5)^3}$ 

\includegraphics[scale=0.5]{start.pdf}
{{$\frac{2x}{(3x-5)^3}-\frac{9(x^2-3)}{(3x-5)^4}$}}
\includegraphics[scale=0.5]{end.pdf}


\item $f(x) = (x^2+2x)^5(x^2-4x)^3$ 

\includegraphics[scale=0.5]{start.pdf}
{{$5(2x+2)(x^2+2x)^4(x^2-4x)^3+3(2x-4)(x^2-4x)^2(x^2+2x)^5$}}
\includegraphics[scale=0.5]{end.pdf}


\item $f(x) = \sin {\left(\frac{\pi}{x}\right)}$ 

\includegraphics[scale=0.5]{start.pdf}
{{$-\pi x^{-2}\cos{\left(\frac{\pi}{x}\right)}$}}
\includegraphics[scale=0.5]{end.pdf}


\item $f(x) = \sin{(\sin{2x})}$ 

\includegraphics[scale=0.5]{start.pdf}
{{$2\cos{(\sin{2x})}\cos{2x}$}}
\includegraphics[scale=0.5]{end.pdf}


\item $f(x) = \tan^2{(x^2-1)}$ 

\includegraphics[scale=0.5]{start.pdf}
{{$4x\tan{(x^2-1)}\sec^2{(x^2-1)}$}}
\includegraphics[scale=0.5]{end.pdf}


\item $f(x) = \frac{2}{(x^5+4x^3-4x)^3}$ 

\includegraphics[scale=0.5]{start.pdf}
{{$\frac{-6(5x^4+12x^2-4)}{(x^5+4x^3-4x)^4}$}}
\includegraphics[scale=0.5]{end.pdf}


\item $f(x) = \left(\frac{x^2-1}{x^2+1}\right)^3$ 

\includegraphics[scale=0.5]{start.pdf}
{{$\frac{12x(x^2-1)^2}{(x^2+1)^4}$}}
\includegraphics[scale=0.5]{end.pdf}


\item $y=4x^2\csc{5x}$

\includegraphics[scale=0.5]{start.pdf}
{{$8x\csc{(5x)}-20x^2\csc{(5x)}\cot{(5x)}$}}
\includegraphics[scale=0.5]{end.pdf}


\item $y=\tan{(4+x^2\sin{3x})}$

\includegraphics[scale=0.5]{start.pdf}
{{$\left(3x^2\cos{3x}+2x\sin{3x}\right)\sec^2{\left(4+x^2\sin{3x}\right)}$}}
\includegraphics[scale=0.5]{end.pdf}


\item Use the given table to calculate each of the following quantities:

\begin{center}
\begin{tabular}{c|c|c|c|c}
$x$ & $f(x)$ & $f^{\prime}(x)$ & $g(x)$ & $g^{\prime}(x)$\\
\hline
1 & $-2$ & $-5$ & 3 & 9\\
2& 5 & $-3$ & 4 & $-2$\\
3 & $-1$ & 6  & 7 & $-6$\\
4 & 3 & 1 & $-2$ & 5\\
5 & 4 & 7 & 1 & 8
\end{tabular}
\end{center}

\begin{enumerate}

\item $\left.\frac{d}{dx}[f(g(x))]\right|_{x=2}$

\includegraphics[scale=0.5]{start.pdf}
{{$-2$}}
\includegraphics[scale=0.5]{end.pdf}


\item $(f \circ g)^{\prime}(1)$

\includegraphics[scale=0.5]{start.pdf}
{{$54$}}
\includegraphics[scale=0.5]{end.pdf}


\item $\left.\frac{d}{dx}[f(3x)]\right|_{x=1}$

\includegraphics[scale=0.5]{start.pdf}
{{$18$}}
\includegraphics[scale=0.5]{end.pdf}


\item $\left.\frac{d}{dx}\left[g\left(\sqrt{2}\sin{\left(\frac{\pi}{4}x\right)}\right)\right]\right|_{x=3}$

\includegraphics[scale=0.5]{start.pdf}
{{$-\frac{9\pi}{4}$}}
\includegraphics[scale=0.5]{end.pdf}


\item $h^{\prime}(2)$ if $h(x)=x^2f(g(x))$

\includegraphics[scale=0.5]{start.pdf}
{{$4$}}
\includegraphics[scale=0.5]{end.pdf}


\end{enumerate}

\end{enumerate}

\noindent {\bf For problems 18-20, calculate $\frac{d^2y}{dx^2}$.} 

\begin{enumerate}
\setcounter{enumi}{17}

\item $y = \sin{3x}$ 

\includegraphics[scale=0.5]{start.pdf}
{{$-9\sin{3x}$}}
\includegraphics[scale=0.5]{end.pdf}


\item $y = x\left(1+\frac{1}{x}\right)^2$ 

\includegraphics[scale=0.5]{start.pdf}
{{$2x^{-3}$}}
\includegraphics[scale=0.5]{end.pdf}


\item $y = \frac{1}{1-2x}$ 

\includegraphics[scale=0.5]{start.pdf}
{{$\frac{8}{(1-2x)^3}$}}
\includegraphics[scale=0.5]{end.pdf}


\item  Suppose that $f(x)$ is a twice differentiable function and define $g(x)=x^3f(2x)$.  Compute $g^{\prime \prime}(x)$ in terms of $f$, $f^{\prime}$, and $f^{\prime \prime}$

\includegraphics[scale=0.5]{start.pdf}
{{$g^{\prime \prime}(x)=4x^3f^{\prime \prime}(2x)+12x^2f^{\prime}(2x)+6xf(2x)$}}
\includegraphics[scale=0.5]{end.pdf}


\item Let $f(x)=\frac{5}{(x^2+1)^3}$.  Compute an equation of the tangent line to the graph of $f(x)$ at $x=0$.

\includegraphics[scale=0.5]{start.pdf}
{{$y=5$}}
\includegraphics[scale=0.5]{end.pdf}


\item Where does the tangent line to $y=(5x+7)^3$ at the point $(-1,8)$ cross the $x$-axis?

\includegraphics[scale=0.5]{start.pdf}
{{$x=-\frac{17}{15}$}}
\includegraphics[scale=0.5]{end.pdf}


\item Find all points on the graph of $y=\sin^2{x}$ where the tangent lines are parallel to the line $y=x$.

\includegraphics[scale=0.5]{start.pdf}
{{$\frac{\pi}{4}+\pi k$ where $k$ is any integer}}
\includegraphics[scale=0.5]{end.pdf}


\item What is the 100th derivative of $y=\sin{(2x)}$?

\includegraphics[scale=0.5]{start.pdf}
{{$2^{100}\sin{2x}$}}
\includegraphics[scale=0.5]{end.pdf}


\item {\bf Multiple Choice:} The derivative of $y=x^2\cos{\left(\frac{1}{x}\right)}$ is

\begin{enumerate}

\item $2x\cos{\left(\frac{1}{x}\right)}-x^2\sin{\left(\frac{1}{x}\right)}$

\item $\frac{2}{x}\sin{\left(\frac{1}{x}\right)}$

\item $-2x\sin{\left(\frac{1}{x}\right)}$

\item $2x\cos{\left(\frac{1}{x}\right)}+\sin{\left(\frac{1}{x}\right)}$

\item $\sin{\left(\frac{1}{x}\right)}$

\end{enumerate}

\includegraphics[scale=0.5]{start.pdf}
{{D}}
\includegraphics[scale=0.5]{end.pdf}


\end{enumerate}

\end{document}